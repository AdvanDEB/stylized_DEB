\documentclass[12pt,a4paper]{article}
\usepackage[utf8]{inputenc}
\usepackage[english]{babel}
\usepackage[T1]{fontenc}
\usepackage{longtable}
\usepackage{array}
\usepackage{booktabs}
\usepackage{geometry}
\usepackage{xcolor}
\usepackage{colortbl}
\usepackage{titling}
\usepackage{adjustbox}

\geometry{margin=2cm}

% Style definitions
\definecolor{lightgray}{gray}{0.9}
\definecolor{darkblue}{RGB}{25,25,112}

\setlength{\parindent}{0pt}

% Adjust title spacing
\setlength{\droptitle}{-3cm}

% Add logo to title
\pretitle{\begin{flushright}\includegraphics[width=2cm]{LOGO_AdvanDEB_V.png}\end{flushright}\vspace{-1cm}\begin{center}\LARGE}
	\posttitle{\end{center}}


% Title page
\title{\textbf{DEB Stylized Facts I}}
\author{}
\date{Heraklion, June 2025}

\begin{document}
	
	\maketitle
	
	\tableofcontents
	\newpage
	
	\small
	
	% =====================================================
	\section{Universal Laws of Biological Organization}
	% =====================================================
	Instructions: \textbf{Accuracy} (1 to 5: \textbf{1} - fact is completely inaccurate; \textbf{5} - fact is completely accurate, additional: \textbf{0} - I have no opinion on the accuracy of this statement); \textbf{I have an explanation} (\textbf{Yes} - I can explain or try to explain the accuracy statement; \textbf{No} - I cannot explain or try to explain the accuracy statement, \textbf{empty cell} - I don't know enough about this statement); \textbf{Importance} (1 to 5: \textbf{1} - least important; \textbf{5} - most important for understanding DEB theory)
	
	\begin{longtable}{|p{0.8cm}|p{7.5cm}|p{2.2cm}|p{2.2cm}|p{2.2cm}|}
		\hline
		\rowcolor{darkblue}
		\textcolor{white}{\textbf{No.}} & \textcolor{white}{\textbf{DEB Stylized Fact}} & \textcolor{white}{\textbf{Accuracy}} & \textcolor{white}{\textbf{I have an explanation}} & \textcolor{white}{\textbf{Importance}} \\
		\hline
		\endfirsthead
		
		\hline
		\rowcolor{darkblue}
		\textcolor{white}{\textbf{No.}} & \textcolor{white}{\textbf{DEB Stylized Fact}} & \textcolor{white}{\textbf{Accuracy}} & \textcolor{white}{\textbf{I have an explanation}} & \textcolor{white}{\textbf{Importance}} \\
		\hline
		\endhead
		
		\hline
		\endfoot
		
		1 & All organisms metabolize energy according to similar laws: there are reserves, structure, and energy flow. & & & \\
		\hline
		2 & Growth and development are discrete phases, where energy is first used for growth and maintenance, and only after reaching a certain level of maturation for reproduction. & & & \\
		\hline
		3 & Reserves and structure have different chemical compositions: reserve is homeostatic (chemically constant), while structure composition can change. & & & \\
		\hline
		4 & All cells have the same reserve density per unit volume when in good condition. & & & \\
		\hline
		5 & Structure maintenance consumes energy proportional to structure mass, not reserves. & & & \\
		\hline
		6 & Synthesis of new structure requires both reserve and energy from reserves for synthetic processes. & & & \\
		\hline
		7 & Metabolic cycles are connected through common metabolites such as ATP, NADH, and CoA derivatives. & & & \\
		\hline
		8 & Specific energy density in reserves is constant across all organism species. & & & \\
		\hline
		9 & Structure has lower energy density than reserves, which explains different roles. & & & \\
		\hline
		10 & All metabolic transformations follow the principle of mass and energy conservation. & & & \\
		\hline
		11 & Energy allocation follows a strict hierarchy: maintenance first, then growth, then reproduction. & & & \\
		\hline
		12 & Reserve dynamics are governed by first-order kinetics with species-specific parameters. & & & \\
		\hline
		13 & Structural mass cannot decrease except through damage or aging processes. & & & \\
		\hline
		14 & Energy conversion efficiency from reserves to structure is species-specific but constant. & & & \\
		\hline
		15 & All organisms use the same fundamental energy currency (ATP) regardless of complexity. & & & \\
		\hline
		16 & Metabolic overhead costs are proportional to metabolic rate across all organisms. & & & \\
		\hline
		17 & Energy storage capacity is limited by physical and chemical constraints of reserve compounds. & & & \\
		\hline
		18 & Maintenance costs increase with structural complexity and specialization. & & & \\
		\hline
		19 & Energy flux through an organism must balance inputs and outputs at steady state. & & & \\
		\hline
		20 & Reserve composition reflects the nutritional history and quality of consumed food. & & & \\
		\hline
		21 & Structural proteins have longer turnover times than metabolic enzymes. & & & \\
		\hline
		22 & Energy mobilization from reserves follows enzyme kinetics and substrate availability. & & & \\
		\hline
		23 & Metabolic pathways are conserved across taxa with similar energy requirements. & & & \\
		\hline
		24 & Cellular energy demand determines the rate of reserve utilization. & & & \\
		\hline
		25 & Structure formation requires specific amino acid compositions and cofactors. & & & \\
		\hline
		26 & Energy losses through heat production are unavoidable in all metabolic processes. & & & \\
		\hline
		27 & Reserve depletion triggers metabolic adjustments to reduce energy expenditure. & & & \\
		\hline
		28 & Structural damage accumulation increases maintenance energy requirements over time. & & & \\
		\hline
		29 & Energy conversion efficiency decreases with increasing metabolic rate. & & & \\
		\hline
		30 & All organisms maintain energy homeostasis through feedback mechanisms. & & & \\
		\hline
		31 & Reserve synthesis requires energy investment beyond the energy content of reserves. & & & \\
		\hline
		32 & Metabolic rate scales with active biomass, not total biomass including reserves. & & & \\
		\hline
		33 & Energy allocation patterns are genetically determined but environmentally modulated. & & & \\
		\hline
		34 & Structural maintenance involves both protein synthesis and damage repair mechanisms. & & & \\
		\hline
		35 & Reserve quality affects energy yield and metabolic efficiency of utilization. & & & \\
		\hline
		36 & Energy flow patterns determine the maximum sustainable growth rate of organisms. & & & \\
		\hline
		37 & Metabolic flexibility allows organisms to switch between different energy sources. & & & \\
		\hline
		38 & Structure-specific maintenance costs vary with tissue type and metabolic activity. & & & \\
		\hline
		39 & Energy storage forms (lipids, carbohydrates, proteins) have distinct mobilization kinetics. & & & \\
		\hline
		40 & Metabolic coordination requires signaling molecules that consume energy themselves. & & & \\
		\hline
		41 & Reserve compartmentalization allows for specialized energy allocation strategies. & & & \\
		\hline
		42 & Energy investment in maintenance prevents structural degradation and functional decline. & & & \\
		\hline
		43 & Metabolic networks exhibit redundancy that provides robustness against perturbations. & & & \\
		\hline
		44 & Structure formation follows template-directed processes requiring energy input. & & & \\
		\hline
		45 & Energy balance determines whether an organism is in anabolic or catabolic state. & & & \\
		\hline
		46 & Reserve dynamics respond to environmental conditions with characteristic time lags. & & & \\
		\hline
		47 & Structural complexity limits the maximum efficiency of energy utilization. & & & \\
		\hline
		48 & Energy allocation trade-offs constrain simultaneous investment in multiple functions. & & & \\
		\hline
		49 & Metabolic rate adjustment mechanisms operate on different temporal scales. & & & \\
		\hline
		50 & Reserve mobilization patterns reflect the energetic priorities of different life stages. & & & \\
		\hline
		51 & Energy conversion steps involve intermediate metabolites with specific functions. & & & \\
		\hline
		52 & Structural maintenance requirements increase with environmental stress exposure. & & & \\
		\hline
		53 & Reserve synthesis rates are limited by enzyme capacity and substrate availability. & & & \\
		\hline
		54 & Energy allocation efficiency improves with optimal nutrient ratios in food. & & & \\
		\hline
		55 & Metabolic pathways exhibit saturation kinetics at high substrate concentrations. & & & \\
		\hline
		56 & Structure formation requires coordination between multiple biosynthetic pathways. & & & \\
		\hline
		57 & Energy storage capacity scales with organism size but not proportionally. & & & \\
		\hline
		58 & Reserve composition changes systematically during different metabolic states. & & & \\
		\hline
		59 & Maintenance energy allocation prevents accumulation of cellular damage. & & & \\
		\hline
		60 & Energy flow patterns determine the temporal dynamics of growth and development. & & & \\
		\hline
		61 & Metabolic efficiency varies with the complexity of biosynthetic requirements. & & & \\
		\hline
		62 & Reserve utilization patterns reflect genetic programming and environmental adaptation. & & & \\
		\hline
		63 & Energy conversion processes generate byproducts that require disposal or recycling. & & & \\
		\hline
		64 & Structural maintenance involves quality control mechanisms that consume energy. & & & \\
		\hline
		65 & Reserve mobilization rates adapt to match energy demand patterns. & & & \\
		\hline
		66 & Energy allocation strategies reflect evolutionary optimization under constraints. & & & \\
		\hline
		67 & Metabolic coordination requires information transfer that has energetic costs. & & & \\
		\hline
		68 & Structure formation efficiency depends on the availability of building blocks. & & & \\
		\hline
		69 & Energy balance fluctuations trigger homeostatic responses to maintain stability. & & & \\
		\hline
		70 & Reserve dynamics exhibit hysteresis effects due to storage and mobilization kinetics. & & & \\
		\hline
		71 & Metabolic network topology determines energy flow patterns and efficiency. & & & \\
		\hline
		72 & Structural maintenance costs increase with the degree of cellular specialization. & & & \\
		\hline
		73 & Energy allocation patterns change systematically throughout the life cycle. & & & \\
		\hline
		74 & Reserve composition reflects adaptive responses to environmental variability. & & & \\
		\hline
		75 & Maintenance energy requirements scale with the complexity of regulatory systems. & & & \\
		\hline
		76 & Energy conversion efficiency is constrained by thermodynamic limits. & & & \\
		\hline
		77 & Metabolic flexibility requires investment in multiple enzyme systems. & & & \\
		\hline
		78 & Structural integrity maintenance becomes more costly with increasing organism age. & & & \\
		\hline
		79 & Energy flow coordination involves both positive and negative feedback mechanisms. & & & \\
		\hline
		80 & Reserve quality control mechanisms prevent accumulation of damaged storage compounds. & & & \\
		\hline
		81 & Metabolic rate regulation involves multiple hierarchical control levels. & & & \\
		\hline
		82 & Energy allocation priorities shift in response to external and internal signals. & & & \\
		\hline
		83 & Structure formation follows developmental programs that require precise energy timing. & & & \\
		\hline
		84 & Reserve mobilization efficiency decreases with increasing storage duration. & & & \\
		\hline
		85 & Energy conversion processes exhibit temperature-dependent kinetics. & & & \\
		\hline
		86 & Metabolic coordination requires sensing mechanisms that consume energy resources. & & & \\
		\hline
		87 & Structural maintenance involves both preventive and corrective energy investments. & & & \\
		\hline
		88 & Energy balance stability requires buffering mechanisms against fluctuations. & & & \\
		\hline
		89 & Reserve compartmentalization enables specialized metabolic functions. & & & \\
		\hline
		90 & Maintenance costs include energy for protein folding and conformational stability. & & & \\
		\hline
		91 & Energy allocation efficiency depends on the coordination of metabolic pathways. & & & \\
		\hline
		92 & Metabolic network robustness requires energy investment in redundant pathways. & & & \\
		\hline
		93 & Structure formation involves quality control that rejects defective components. & & & \\
		\hline
		94 & Energy storage strategies reflect trade-offs between capacity and mobilization speed. & & & \\
		\hline
		95 & Reserve utilization patterns optimize energy yield under varying conditions. & & & \\
		\hline
		96 & Metabolic coordination involves hierarchical control systems with energy costs. & & & \\
		\hline
		97 & Structural maintenance requirements vary with tissue function and activity levels. & & & \\
		\hline
		98 & Energy flow patterns determine the sustainability of metabolic strategies. & & & \\
		\hline
		99 & Reserve dynamics exhibit species-specific patterns reflecting ecological adaptations. & & & \\
		\hline
		100 & Energy allocation optimization involves trade-offs between current and future benefits. & & & \\
		\hline
	\end{longtable}
	
	\newpage
	% =====================================================
	\section{Ecological-Physiological Proportions and Relationships}
	% =====================================================
	Instructions: \textbf{Accuracy} (1 to 5: \textbf{1} - fact is completely inaccurate; \textbf{5} - fact is completely accurate, additional: \textbf{0} - I have no opinion on the accuracy of this statement); \textbf{I have an explanation} (\textbf{Yes} - I can explain or try to explain the accuracy statement; \textbf{No} - I cannot explain or try to explain the accuracy statement, \textbf{empty cell} - I don't know enough about this statement); \textbf{Importance} (1 to 5: \textbf{1} - least important; \textbf{5} - most important for understanding DEB theory)
	
	\begin{longtable}{|p{0.8cm}|p{7.5cm}|p{2.2cm}|p{2.2cm}|p{2.2cm}|}
		\hline
		\rowcolor{darkblue}
		\textcolor{white}{\textbf{No.}} & \textcolor{white}{\textbf{DEB Stylized Fact}} & \textcolor{white}{\textbf{Accuracy}} & \textcolor{white}{\textbf{I have an explanation}} & \textcolor{white}{\textbf{Importance}} \\
		\hline
		\endfirsthead
		
		\hline
		\rowcolor{darkblue}
		\textcolor{white}{\textbf{No.}} & \textcolor{white}{\textbf{DEB Stylized Fact}} & \textcolor{white}{\textbf{Accuracy}} & \textcolor{white}{\textbf{I have an explanation}} & \textcolor{white}{\textbf{Importance}} \\
		\hline
		\endhead
		
		\hline
		\endfoot
		
		101 & Metabolic rate increases with body mass, but not proportionally – it follows an allometric relationship (typically scales as $\propto$ mass$^{3/4}$). & & & \\
		\hline
		102 & There is a relationship between body mass and lifespan: larger animals grow slower and live longer. & & & \\
		\hline
		103 & Food intake capacity is limited by body surface area, while maintenance needs are proportional to volume. & & & \\
		\hline
		104 & Time to puberty and total lifespan increase with body mass – so-called life-history invariants. & & & \\
		\hline
		105 & Maximum assimilation rate scales with body mass to a power less than 1. & & & \\
		\hline
		106 & Maintenance rate per unit structure is constant within species, but varies between species. & & & \\
		\hline
		107 & Reproductive output scales with body mass through the same power as basal metabolism. & & & \\
		\hline
		108 & Assimilation efficiency decreases with size due to geometric constraints. & & & \\
		\hline
		109 & Surface area-to-volume ratio determines metabolic constraints for material transport and heat. & & & \\
		\hline
		110 & Allometric exponent for heart rate is negative ($\sim$-0.25). & & & \\
		\hline
		111 & Breathing rate scales negatively with body mass across species. & & & \\
		\hline
		112 & Blood circulation time increases with body size due to longer transport distances. & & & \\
		\hline
		113 & Metabolic scope (difference between maximum and resting metabolic rate) scales with body mass. & & & \\
		\hline
		114 & Digestive tract length scales with body mass and dietary requirements. & & & \\
		\hline
		115 & Brain mass scales differently from body mass, typically with a lower allometric exponent. & & & \\
		\hline
		116 & Kidney filtration rate scales with metabolic rate and body mass. & & & \\
		\hline
		117 & Bone strength scales with body mass to ensure structural integrity. & & & \\
		\hline
		118 & Muscle mass proportion decreases with increasing body size in most taxa. & & & \\
		\hline
		119 & Heat production rate scales with metabolic rate and affects thermoregulatory strategies. & & & \\
		\hline
		120 & Oxygen consumption rate per unit mass decreases with increasing body size. & & & \\
		\hline
		121 & Swimming speed in aquatic animals scales positively with body length. & & & \\
		\hline
		122 & Flight metabolic rate in birds scales differently from terrestrial locomotion costs. & & & \\
		\hline
		123 & Dive duration in aquatic mammals scales with body mass and oxygen storage capacity. & & & \\
		\hline
		124 & Territory size scales with body mass and energy requirements in many species. & & & \\
		\hline
		125 & Foraging range increases with body size to meet elevated energy demands. & & & \\
		\hline
		126 & Daily energy expenditure scales with body mass but varies with activity patterns. & & & \\
		\hline
		127 & Seasonal fat storage capacity scales with body mass and climate adaptations. & & & \\
		\hline
		128 & Hibernation metabolic depression is more pronounced in smaller mammals. & & & \\
		\hline
		129 & Migration distance correlates with body size and energy storage capabilities. & & & \\
		\hline
		130 & Immune system investment scales with body mass and pathogen exposure risk. & & & \\
		\hline
		131 & Wound healing rate decreases with body size due to circulatory constraints. & & & \\
		\hline
		132 & Sensory organ size scales differently from body mass depending on function. & & & \\
		\hline
		133 & Reproductive organ size scales with body mass and mating system characteristics. & & & \\
		\hline
		134 & Gestation time increases with body mass and offspring development requirements. & & & \\
		\hline
		135 & Litter size typically decreases with increasing maternal body mass. & & & \\
		\hline
		136 & Parental care duration scales with offspring development time and body size. & & & \\
		\hline
		137 & Age at first reproduction increases with adult body size across taxa. & & & \\
		\hline
		138 & Maximum population density decreases with average individual body mass. & & & \\
		\hline
		139 & Home range size scales with body mass and resource distribution patterns. & & & \\
		\hline
		140 & Dispersal distance correlates with body size and habitat fragmentation. & & & \\
		\hline
		141 & Predator-prey mass ratios follow predictable scaling relationships. & & & \\
		\hline
		142 & Trophic level position correlates with body size in food webs. & & & \\
		\hline
		143 & Biomass turnover rate decreases with increasing organism size. & & & \\
		\hline
		144 & Population growth rate scales negatively with body mass. & & & \\
		\hline
		145 & Extinction risk correlates with body size through population size effects. & & & \\
		\hline
		146 & Metabolic cold adaptation involves changes in allometric relationships. & & & \\
		\hline
		147 & Altitude adaptation affects metabolic scaling due to oxygen availability. & & & \\
		\hline
		148 & Aquatic vs terrestrial environments impose different scaling constraints. & & & \\
		\hline
		149 & Developmental temperature affects adult allometric relationships. & & & \\
		\hline
		150 & Sexual dimorphism in body size affects metabolic scaling within species. & & & \\
		\hline
		151 & Seasonal changes in body mass alter temporary allometric relationships. & & & \\
		\hline
		152 & Aging affects metabolic scaling through changes in body composition. & & & \\
		\hline
		153 & Diet quality influences the scaling of digestive system components. & & & \\
		\hline
		154 & Social organization affects metabolic scaling through cooperative behaviors. & & & \\
		\hline
		155 & Parasitism alters host metabolic scaling relationships. & & & \\
		\hline
		156 & Symbiotic relationships modify allometric patterns in partner organisms. & & & \\
		\hline
		157 & Habitat complexity influences locomotor scaling relationships. & & & \\
		\hline
		158 & Predation pressure affects the evolution of size-related traits and scaling. & & & \\
		\hline
		159 & Competition intensity scales with body size overlap between species. & & & \\
		\hline
		160 & Resource availability influences optimal body size and scaling patterns. & & & \\
		\hline
		161 & Climate variability affects size-related adaptations and allometric relationships. & & & \\
		\hline
		162 & Island environments often lead to altered scaling relationships. & & & \\
		\hline
		163 & Urbanization impacts metabolic scaling through environmental changes. & & & \\
		\hline
		164 & Pollution exposure affects size-dependent physiological processes. & & & \\
		\hline
		165 & Conservation status correlates with body size through multiple mechanisms. & & & \\
		\hline
		166 & Phylogenetic constraints influence the evolution of allometric relationships. & & & \\
		\hline
		167 & Developmental plasticity allows modification of scaling relationships. & & & \\
		\hline
		168 & Life history strategies reflect different solutions to scaling constraints. & & & \\
		\hline
		169 & Biomechanical limits impose constraints on size-related performance scaling. & & & \\
		\hline
		170 & Physiological limits determine maximum achievable body sizes in different taxa. & & & \\
		\hline
		171 & Cellular size constraints influence tissue-level scaling relationships. & & & \\
		\hline
		172 & Molecular processes scale differently from organ-level functions. & & & \\
		\hline
		173 & Network topology affects scaling of circulatory and nervous systems. & & & \\
		\hline
		174 & Scaling relationships differ between growth phases and adult maintenance. & & & \\
		\hline
		175 & Environmental stress modifies normal allometric relationships. & & & \\
		\hline
		176 & Reproductive cycles influence temporary deviations from scaling laws. & & & \\
		\hline
		177 & Behavioral adaptations can compensate for scaling-imposed constraints. & & & \\
		\hline
		178 & Technological analogies help understand biological scaling principles. & & & \\
		\hline
		179 & Mathematical models predict scaling relationships from first principles. & & & \\
		\hline
		180 & Fractal geometry explains some aspects of biological scaling patterns. & & & \\
		\hline
		181 & Optimization theory provides frameworks for understanding scaling evolution. & & & \\
		\hline
		182 & Scaling relationships have predictive power for ecological and evolutionary processes. & & & \\
		\hline
		183 & Cross-species comparisons reveal universal scaling principles. & & & \\
		\hline
		184 & Intraspecific scaling may differ from interspecific patterns. & & & \\
		\hline
		185 & Ontogenetic scaling changes throughout individual development. & & & \\
		\hline
		186 & Scaling violations often indicate underlying physiological constraints or adaptations. & & & \\
		\hline
		187 & Metabolic scaling forms the foundation for understanding ecological energetics. & & & \\
		\hline
		188 & Size-dependent mortality patterns influence population dynamics and evolution. & & & \\
		\hline
		189 & Scaling relationships provide tools for comparative physiology and ecology. & & & \\
		\hline
		190 & Body size evolution reflects optimization under multiple scaling constraints. & & & \\
		\hline
		191 & Scaling laws connect individual physiology to ecosystem-level processes. & & & \\
		\hline
		192 & Allometric relationships facilitate predictions about unstudied species. & & & \\
		\hline
		193 & Scaling analysis reveals the importance of size in biological organization. & & & \\
		\hline
		194 & Size-based approaches provide frameworks for conservation biology applications. & & & \\
		\hline
		195 & Metabolic scaling influences species interactions and community structure. & & & \\
		\hline
		196 & Body size distributions reflect ecological and evolutionary processes. & & & \\
		\hline
		197 & Scaling relationships help predict responses to environmental change. & & & \\
		\hline
		198 & Size-dependent processes link ecology and evolution across biological scales. & & & \\
		\hline
		199 & Allometric analysis provides insights into functional design principles. & & & \\
		\hline
		200 & Scaling laws represent fundamental constraints on biological organization and function. & & & \\
		\hline
	\end{longtable}
	
	% =====================================================
	\section{Temporal Characteristics of Development and Growth}
	% =====================================================
	Instructions: \textbf{Accuracy} (1 to 5: \textbf{1} - fact is completely inaccurate; \textbf{5} - fact is completely accurate, additional: \textbf{0} - I have no opinion on the accuracy of this statement); \textbf{I have an explanation} (\textbf{Yes} - I can explain or try to explain the accuracy statement; \textbf{No} - I cannot explain or try to explain the accuracy statement, \textbf{empty cell} - I don't know enough about this statement); \textbf{Importance} (1 to 5: \textbf{1} - least important; \textbf{5} - most important for understanding DEB theory)
	
	\begin{longtable}{|p{0.8cm}|p{7.5cm}|p{2.2cm}|p{2.2cm}|p{2.2cm}|}
		\hline
		\rowcolor{darkblue}
		\textcolor{white}{\textbf{No.}} & \textcolor{white}{\textbf{DEB Stylized Fact}} & \textcolor{white}{\textbf{Accuracy}} & \textcolor{white}{\textbf{I have an explanation}} & \textcolor{white}{\textbf{Importance}} \\
		\hline
		\endfirsthead
		
		\hline
		\rowcolor{darkblue}
		\textcolor{white}{\textbf{No.}} & \textcolor{white}{\textbf{DEB Stylized Fact}} & \textcolor{white}{\textbf{Accuracy}} & \textcolor{white}{\textbf{I have an explanation}} & \textcolor{white}{\textbf{Importance}} \\
		\hline
		\endhead
		
		\hline
		\endfoot
		
		201 & Development and growth occur simultaneously but independently – development is a function of accumulated maturation energy (E$_H$), while growth is a function of available energy for somatic structure. & & & \\
		\hline
		202 & Puberty occurs when a certain amount of maturation energy (E$_H^p$) is reached, regardless of body size. & & & \\
		\hline
		203 & Growth often slows after puberty because increasing energy is directed toward reproduction. & & & \\
		\hline
		204 & Von Bertalanffy growth is an emergent property of DEB dynamics, not a fundamental law. & & & \\
		\hline
		205 & Specific growth rate decreases exponentially with size before puberty. & & & \\
		\hline
		206 & Mass doubling time scales positively with adult size. & & & \\
		\hline
		207 & Embryogenesis follows predictable patterns of reserve expenditure and maturation energy accumulation. & & & \\
		\hline
		208 & Juvenile period lasts longer in species with larger adult size. & & & \\
		\hline
		209 & There is an optimal size for transitioning to reproduction that depends on mortality and resources. & & & \\
		\hline
		210 & Aging begins when reserves can no longer maintain metabolic needs. & & & \\
		\hline
		211 & Developmental stage transitions are triggered by specific maturation energy thresholds. & & & \\
		\hline
		212 & Growth rate acceleration occurs during early juvenile phases when maintenance costs are low. & & & \\
		\hline
		213 & Metamorphosis involves major reorganization of energy allocation patterns. & & & \\
		\hline
		214 & Indeterminate growth continues throughout life in many species with low maintenance costs. & & & \\
		\hline
		215 & Growth efficiency decreases with age due to increasing maintenance requirements. & & & \\
		\hline
		216 & Maturation energy accumulation rate depends on surplus energy after maintenance and growth. & & & \\
		\hline
		217 & Developmental timing is more sensitive to temperature than growth rate. & & & \\
		\hline
		218 & Growth compensation occurs after periods of food restriction through increased allocation efficiency. & & & \\
		\hline
		219 & Sexual size dimorphism emerges through different growth trajectories after puberty. & & & \\
		\hline
		220 & Larval development involves distinct energy allocation strategies compared to adult stages. & & & \\
		\hline
		221 & Growth cessation occurs when maintenance costs equal energy intake capacity. & & & \\
		\hline
		222 & Developmental plasticity allows adjustment of maturation timing based on environmental conditions. & & & \\
		\hline
		223 & Growth spurts coincide with periods of high food availability and low stress. & & & \\
		\hline
		224 & Senescence involves progressive decline in maintenance efficiency and growth capacity. & & & \\
		\hline
		225 & Juvenile survival probability increases with faster early growth rates. & & & \\
		\hline
		226 & Developmental synchrony within populations reflects similar environmental triggers. & & & \\
		\hline
		227 & Growth trajectory reversibility is limited by irreversible developmental commitments. & & & \\
		\hline
		228 & Maximum growth rate occurs during optimal environmental conditions and body size. & & & \\
		\hline
		229 & Developmental heterochrony leads to evolutionary changes in life history timing. & & & \\
		\hline
		230 & Growth allometry changes during development as different body parts grow at different rates. & & & \\
		\hline
		231 & Maturation investment competes with structural growth for available energy resources. & & & \\
		\hline
		232 & Developmental constraints limit the range of possible growth trajectories. & & & \\
		\hline
		233 & Growth resumption after dormancy follows predictable patterns of energy reallocation. & & & \\
		\hline
		234 & Age-specific growth rates reflect changing energy allocation priorities throughout life. & & & \\
		\hline
		235 & Developmental modularity allows independent timing of different organ system maturation. & & & \\
		\hline
		236 & Growth plateau duration varies with species-specific maintenance cost patterns. & & & \\
		\hline
		237 & Maturation acceleration occurs under high mortality risk to ensure reproductive success. & & & \\
		\hline
		238 & Growth variability within cohorts reflects individual differences in energy allocation efficiency. & & & \\
		\hline
		239 & Developmental checkpoints ensure proper maturation energy accumulation before stage transitions. & & & \\
		\hline
		240 & Growth asymptote reflects the balance between energy intake capacity and maintenance costs. & & & \\
		\hline
		241 & Maturation energy requirements scale with adult size and reproductive investment needs. & & & \\
		\hline
		242 & Growth pattern inheritance involves both genetic and epigenetic factors affecting energy allocation. & & & \\
		\hline
		243 & Developmental windows exist for critical organ formation requiring specific energy allocation timing. & & & \\
		\hline
		244 & Growth hormone sensitivity changes during development affecting energy utilization efficiency. & & & \\
		\hline
		245 & Maturation timing optimization balances current survival against future reproductive potential. & & & \\
		\hline
		246 & Growth cessation signals redirect available energy toward maintenance and reproduction preparation. & & & \\
		\hline
		247 & Developmental programming during early life affects adult growth capacity and maturation patterns. & & & \\
		\hline
		248 & Growth recovery after stress depends on reserve availability and maintenance debt accumulation. & & & \\
		\hline
		249 & Maturation energy thresholds can be modified by environmental conditions and stress exposure. & & & \\
		\hline
		250 & Growth trajectory prediction requires knowledge of both genetic potential and environmental constraints. & & & \\
		\hline
		251 & Developmental gene expression patterns reflect underlying energy allocation priorities. & & & \\
		\hline
		252 & Growth rate optimization involves trade-offs between speed and efficiency of energy utilization. & & & \\
		\hline
		253 & Maturation timing heritability suggests genetic control of energy allocation developmental programs. & & & \\
		\hline
		254 & Growth pattern disruption can have cascading effects on later developmental stages. & & & \\
		\hline
		255 & Developmental stability requires consistent energy allocation patterns throughout growth phases. & & & \\
		\hline
		256 & Growth measurement techniques must account for changes in body composition during development. & & & \\
		\hline
		257 & Maturation biomarkers reflect accumulated developmental energy investment and stage transitions. & & & \\
		\hline
		258 & Growth modeling requires integration of both somatic and reproductive energy allocation patterns. & & & \\
		\hline
		259 & Developmental timing coordination ensures optimal resource utilization across multiple organ systems. & & & \\
		\hline
		260 & Growth cessation mechanisms prevent excessive energy investment in somatic tissue maintenance. & & & \\
		\hline
		261 & Maturation energy investment represents irreversible commitment to reproductive capability development. & & & \\
		\hline
		262 & Growth pattern analysis reveals species-specific energy allocation strategies and ecological adaptations. & & & \\
		\hline
		263 & Developmental temperature effects persist throughout life affecting adult energy allocation efficiency. & & & \\
		\hline
		264 & Growth trajectory modifications require understanding of underlying DEB parameter relationships. & & & \\
		\hline
		265 & Maturation timing flexibility provides adaptive responses to environmental variability and uncertainty. & & & \\
		\hline
		266 & Growth efficiency measurement must consider both mass gain and structural quality improvements. & & & \\
		\hline
		267 & Developmental energy budgets constrain the simultaneous investment in growth and maturation processes. & & & \\
		\hline
		268 & Growth pattern conservation across related species suggests fundamental energy allocation constraints. & & & \\
		\hline
		269 & Maturation energy accumulation rate determines the minimum time required for reproductive development. & & & \\
		\hline
		270 & Growth cessation timing affects lifetime reproductive output through energy reallocation patterns. & & & \\
		\hline
		271 & Developmental robustness mechanisms ensure proper maturation despite environmental perturbations. & & & \\
		\hline
		272 & Growth rate seasonal variation reflects cyclical changes in energy availability and allocation priorities. & & & \\
		\hline
		273 & Maturation investment scaling ensures appropriate reproductive capacity relative to adult body size. & & & \\
		\hline
		274 & Growth pattern disruption recovery depends on developmental stage and available energy reserves. & & & \\
		\hline
		275 & Developmental coordination mechanisms synchronize growth and maturation with environmental cycles. & & & \\
		\hline
		276 & Growth trajectory analysis provides insights into past environmental conditions and energy availability. & & & \\
		\hline
		277 & Maturation timing optimization reflects evolutionary responses to mortality patterns and resource availability. & & & \\
		\hline
		278 & Growth measurement standardization requires accounting for species-specific developmental patterns. & & & \\
		\hline
		279 & Developmental energy allocation efficiency affects competitive ability and survival probability. & & & \\
		\hline
		280 & Growth pattern predictability enables forecasting of individual and population development trajectories. & & & \\
		\hline
		281 & Maturation energy threshold modification provides adaptive flexibility in variable environments. & & & \\
		\hline
		282 & Growth cessation reversibility is limited by developmental constraints and energy allocation commitments. & & & \\
		\hline
		283 & Developmental timing synchronization ensures optimal energy utilization across life history stages. & & & \\
		\hline
		284 & Growth efficiency optimization involves balancing speed against long-term maintenance requirements. & & & \\
		\hline
		285 & Maturation investment protection mechanisms prevent energy reallocation during critical developmental periods. & & & \\
		\hline
		286 & Growth trajectory heritability reflects genetic control of energy allocation developmental programs. & & & \\
		\hline
		287 & Developmental plasticity limits enable adaptive responses while maintaining essential growth patterns. & & & \\
		\hline
		288 & Growth pattern conservation suggests fundamental constraints on energy allocation evolution. & & & \\
		\hline
		289 & Maturation energy requirements increase with reproductive complexity and parental investment strategies. & & & \\
		\hline
		290 & Growth cessation mechanisms prevent excessive investment in maintenance-costly somatic structures. & & & \\
		\hline
		291 & Developmental robustness trade-offs involve costs of maintaining flexibility against specialized efficiency. & & & \\
		\hline
		292 & Growth rate optimization requires balancing current energy utilization against future allocation needs. & & & \\
		\hline
		293 & Maturation timing coordination ensures reproductive readiness coincides with optimal environmental conditions. & & & \\
		\hline
		294 & Growth measurement precision affects accuracy of energy allocation analysis and developmental predictions. & & & \\
		\hline
		295 & Developmental energy investment irreversibility creates constraints on life history evolution. & & & \\
		\hline
		296 & Growth pattern analysis reveals adaptation to specific ecological niches and resource utilization strategies. & & & \\
		\hline
		297 & Maturation energy accumulation monitoring provides early indicators of reproductive development progress. & & & \\
		\hline
		298 & Growth trajectory modification potential decreases with advancing developmental stage and age. & & & \\
		\hline
		299 & Developmental coordination complexity increases with organism complexity and life history strategies. & & & \\
		\hline
		300 & Growth and maturation integration represents fundamental principle of DEB theory applications. & & & \\
		\hline
	\end{longtable}
	

\newpage
% =====================================================
\section{Reproductive Strategy}
% =====================================================
Instructions: \textbf{Accuracy} (1 to 5: \textbf{1} - fact is completely inaccurate; \textbf{5} - fact is completely accurate, additional: \textbf{0} - I have no opinion on the accuracy of this statement); \textbf{I have an explanation} (\textbf{Yes} - I can explain or try to explain the accuracy statement; \textbf{No} - I cannot explain or try to explain the accuracy statement, \textbf{empty cell} - I don't know enough about this statement); \textbf{Importance} (1 to 5: \textbf{1} - least important; \textbf{5} - most important for understanding DEB theory)

\begin{longtable}{|p{0.8cm}|p{7.5cm}|p{2.2cm}|p{2.2cm}|p{2.2cm}|}
	\hline
	\rowcolor{darkblue}
	\textcolor{white}{\textbf{No.}} & \textcolor{white}{\textbf{DEB Stylized Fact}} & \textcolor{white}{\textbf{Accuracy}} & \textcolor{white}{\textbf{I have an explanation}} & \textcolor{white}{\textbf{Importance}} \\
	\hline
	\endfirsthead
	
	\hline
	\rowcolor{darkblue}
	\textcolor{white}{\textbf{No.}} & \textcolor{white}{\textbf{DEB Stylized Fact}} & \textcolor{white}{\textbf{Accuracy}} & \textcolor{white}{\textbf{I have an explanation}} & \textcolor{white}{\textbf{Importance}} \\
	\hline
	\endhead
	
	\hline
	\endfoot
	
	301 & Energy investment in reproduction is proportional to adult body size, not number of offspring. & & & \\
	\hline
	302 & All species have a trade-off between offspring size and number (few large or many small). & & & \\
	\hline
	303 & Egg cells or embryos have a certain amount of initial reserve (E$_0$), independent of environmental conditions. & & & \\
	\hline
	304 & Reproductive efficiency declines with age due to accumulation of damage. & & & \\
	\hline
	305 & Investment in offspring is optimized according to expected reproductive value. & & & \\
	\hline
	306 & Seasonal reproduction is linked to cyclical resources and energy capacity. & & & \\
	\hline
	307 & Sexual maturation depends on accumulated maturation energy, not chronological age. & & & \\
	\hline
	308 & Parental investment continues after laying/birth through lactation or care. & & & \\
	\hline
	309 & Clutch or litter size is optimized according to available resources and mortality. & & & \\
	\hline
	310 & Iterative vs. semelparous reproductive strategies reflect different life cycles. & & & \\
	\hline
	311 & Reproductive allocation increases with body size but plateaus at maximum energy availability. & & & \\
	\hline
	312 & Gamete production costs scale with reproductive output and offspring size requirements. & & & \\
	\hline
	313 & Mating effort competes with reproductive energy allocation for available resources. & & & \\
	\hline
	314 & Reproductive timing optimization balances offspring survival against parental survival costs. & & & \\
	\hline
	315 & Sex allocation strategies reflect energy investment trade-offs between male and female offspring. & & & \\
	\hline
	316 & Reproductive senescence results from declining energy allocation efficiency with age. & & & \\
	\hline
	317 & Breeding frequency depends on recovery time between reproductive efforts and energy accumulation. & & & \\
	\hline
	318 & Offspring provisioning strategies vary with environmental predictability and resource availability. & & & \\
	\hline
	319 & Reproductive investment flexibility allows adjustment to current environmental conditions. & & & \\
	\hline
	320 & Mate choice affects reproductive energy allocation through selection for optimal partners. & & & \\
	\hline
	321 & Reproductive success depends on both energy allocation quantity and timing optimization. & & & \\
	\hline
	322 & Parental care duration reflects trade-offs between offspring quality and quantity. & & & \\
	\hline
	323 & Reproductive lifespan correlates with maintenance efficiency and damage accumulation rates. & & & \\
	\hline
	324 & Breeding season length affects total reproductive energy allocation and offspring production. & & & \\
	\hline
	325 & Reproductive storage mechanisms enable reproduction during periods of low energy intake. & & & \\
	\hline
	326 & Offspring size optimization depends on juvenile survival probability and growth requirements. & & & \\
	\hline
	327 & Reproductive competition intensity affects energy allocation between reproduction and survival. & & & \\
	\hline
	328 & Sexual dimorphism in energy allocation reflects different reproductive strategies between sexes. & & & \\
	\hline
	329 & Reproductive skipping occurs when energy reserves are insufficient for successful reproduction. & & & \\
	\hline
	330 & Nest building and territory defense costs must be included in total reproductive energy budgets. & & & \\
	\hline
	331 & Reproductive hormone production requires energy allocation that competes with other functions. & & & \\
	\hline
	332 & Multiple mating costs include energy for mate searching, competition, and copulation. & & & \\
	\hline
	333 & Reproductive organ development and maintenance require continuous energy investment. & & & \\
	\hline
	334 & Spawning migration costs represent significant reproductive energy expenditure in many species. & & & \\
	\hline
	335 & Reproductive output variability reflects fluctuations in energy availability and allocation efficiency. & & & \\
	\hline
	336 & Alternative reproductive strategies emerge from different energy allocation optimization solutions. & & & \\
	\hline
	337 & Reproductive investment protection mechanisms prevent energy reallocation during critical periods. & & & \\
	\hline
	338 & Social reproductive systems affect individual energy allocation through cooperative and competitive interactions. & & & \\
	\hline
	339 & Reproductive timing cues integrate environmental signals with internal energy state assessment. & & & \\
	\hline
	340 & Maternal effect investments transfer energy reserves to offspring through various provisioning mechanisms. & & & \\
	\hline
	341 & Reproductive effort scaling ensures appropriate investment relative to body size and condition. & & & \\
	\hline
	342 & Paternal investment strategies reflect species-specific energy allocation patterns and mating systems. & & & \\
	\hline
	343 & Reproductive failure recovery requires energy reallocation to restore reproductive capacity. & & & \\
	\hline
	344 & Breeding habitat quality affects reproductive energy allocation efficiency and success rates. & & & \\
	\hline
	345 & Reproductive synchrony within populations optimizes resource utilization and offspring survival. & & & \\
	\hline
	346 & Age-specific reproductive investment reflects changing survival probability and future reproductive potential. & & & \\
	\hline
	347 & Reproductive technology utilization by organisms includes tools and structures requiring energy investment. & & & \\
	\hline
	348 & Environmental reproductive cues trigger energy reallocation from growth and maintenance to reproduction. & & & \\
	\hline
	349 & Reproductive conflict resolution mechanisms determine energy allocation in social breeding systems. & & & \\
	\hline
	350 & Multiple reproductive cycles within seasons require efficient energy allocation and recovery mechanisms. & & & \\
	\hline
	351 & Reproductive investment heritability suggests genetic control of energy allocation reproductive strategies. & & & \\
	\hline
	352 & Cross-generational reproductive effects involve energy transfer patterns affecting offspring reproductive potential. & & & \\
	\hline
	353 & Reproductive system complexity increases energy allocation requirements for maintenance and function. & & & \\
	\hline
	354 & Environmental reproductive stress affects energy allocation efficiency and reproductive output quality. & & & \\
	\hline
	355 & Reproductive behavior energy costs include courtship, mate guarding, and territorial defense activities. & & & \\
	\hline
	356 & Adaptive reproductive plasticity allows energy allocation adjustment to environmental variability. & & & \\
	\hline
	357 & Reproductive biomarker development reflects underlying energy allocation patterns and reproductive state. & & & \\
	\hline
	358 & Lifetime reproductive success optimization involves energy allocation trade-offs across multiple breeding attempts. & & & \\
	\hline
	359 & Reproductive isolation mechanisms require energy investment that competes with reproductive output. & & & \\
	\hline
	360 & Environmental reproductive signals synchronize energy allocation with optimal breeding conditions. & & & \\
	\hline
	361 & Reproductive energy storage efficiency affects breeding frequency and reproductive lifespan. & & & \\
	\hline
	362 & Cooperative breeding systems redistribute reproductive energy allocation among group members. & & & \\
	\hline
	363 & Reproductive phenology shifts reflect adaptive responses to changing energy availability patterns. & & & \\
	\hline
	364 & Reproductive success measurement requires accounting for energy allocation efficiency and offspring quality. & & & \\
	\hline
	365 & Alternative mating strategies involve different energy allocation patterns between reproductive tactics. & & & \\
	\hline
	366 & Reproductive system evolution reflects optimization of energy allocation under specific ecological constraints. & & & \\
	\hline
	367 & Post-reproductive survival in some species requires continued energy allocation for offspring success. & & & \\
	\hline
	368 & Reproductive energy allocation modeling requires integration of multiple competing demands and constraints. & & & \\
	\hline
	369 & Environmental reproductive disruption affects energy allocation patterns and reproductive timing. & & & \\
	\hline
	370 & Reproductive investment recovery time affects breeding frequency and lifetime reproductive output. & & & \\
	\hline
	371 & Sex-specific reproductive costs create different energy allocation optimization solutions for males and females. & & & \\
	\hline
	372 & Reproductive habitat selection involves energy costs that affect total reproductive energy budgets. & & & \\
	\hline
	373 & Communal reproductive strategies share energy allocation costs among participating individuals. & & & \\
	\hline
	374 & Reproductive timing precision affects energy allocation efficiency and offspring survival probability. & & & \\
	\hline
	375 & Induced reproductive strategies respond to environmental cues with rapid energy allocation shifts. & & & \\
	\hline
	376 & Reproductive conflict resolution requires energy allocation arbitration between competing reproductive interests. & & & \\
	\hline
	377 & Reproductive success prediction requires understanding of energy allocation constraints and optimization strategies. & & & \\
	\hline
	378 & Environmental reproductive change adaptation involves modification of energy allocation reproductive strategies. & & & \\
	\hline
	379 & Reproductive efficiency improvement mechanisms optimize energy allocation for maximum reproductive output. & & & \\
	\hline
	380 & Cross-species reproductive comparisons reveal universal energy allocation principles and species-specific adaptations. & & & \\
	\hline
	381 & Reproductive system maintenance costs increase with reproductive complexity and specialization. & & & \\
	\hline
	382 & Environmental reproductive protection requires energy allocation for reproductive system defense and preservation. & & & \\
	\hline
	383 & Reproductive energy allocation coordination ensures optimal timing and investment across reproductive functions. & & & \\
	\hline
	384 & Climate reproductive adaptation involves modification of energy allocation patterns to match changing conditions. & & & \\
	\hline
	385 & Reproductive success optimization requires balancing energy allocation between current and future reproductive attempts. & & & \\
	\hline
	386 & Environmental reproductive monitoring provides early indicators of energy allocation disruption and reproductive stress. & & & \\
	\hline
	387 & Reproductive strategy evolution reflects long-term optimization of energy allocation under varying environmental pressures. & & & \\
	\hline
	388 & Reproductive energy allocation efficiency affects population growth rate and evolutionary fitness. & & & \\
	\hline
	389 & Human reproductive intervention affects natural energy allocation patterns and reproductive success. & & & \\
	\hline
	390 & Reproductive research applications require understanding of energy allocation principles and species-specific patterns. & & & \\
	\hline
	391 & Conservation reproductive management involves manipulating energy allocation to enhance reproductive success. & & & \\
	\hline
	392 & Reproductive health assessment requires evaluation of energy allocation efficiency and reproductive system function. & & & \\
	\hline
	393 & Technological reproductive assistance affects natural energy allocation patterns in managed populations. & & & \\
	\hline
	394 & Reproductive ecology understanding requires integration of energy allocation principles with environmental interactions. & & & \\
	\hline
	395 & Future reproductive research directions include advanced energy allocation modeling and prediction techniques. & & & \\
	\hline
	396 & Reproductive system conservation requires protection of energy allocation patterns and reproductive habitat quality. & & & \\
	\hline
	397 & Global reproductive change monitoring tracks alterations in energy allocation patterns and reproductive success. & & & \\
	\hline
	398 & Reproductive education programs should include energy allocation principles and reproductive strategy understanding. & & & \\
	\hline
	399 & Reproductive policy development requires consideration of energy allocation constraints and reproductive biology principles. & & & \\
	\hline
	400 & Integrated reproductive science combines energy allocation theory with empirical reproductive biology research. & & & \\
	\hline
\end{longtable}

\newpage
% =====================================================
\section{Metabolism and Environment}
% =====================================================
Instructions: \textbf{Accuracy} (1 to 5: \textbf{1} - fact is completely inaccurate; \textbf{5} - fact is completely accurate, additional: \textbf{0} - I have no opinion on the accuracy of this statement); \textbf{I have an explanation} (\textbf{Yes} - I can explain or try to explain the accuracy statement; \textbf{No} - I cannot explain or try to explain the accuracy statement, \textbf{empty cell} - I don't know enough about this statement); \textbf{Importance} (1 to 5: \textbf{1} - least important; \textbf{5} - most important for understanding DEB theory)

\begin{longtable}{|p{0.8cm}|p{7.5cm}|p{2.2cm}|p{2.2cm}|p{2.2cm}|}
	\hline
	\rowcolor{darkblue}
	\textcolor{white}{\textbf{No.}} & \textcolor{white}{\textbf{DEB Stylized Fact}} & \textcolor{white}{\textbf{Accuracy}} & \textcolor{white}{\textbf{I have an explanation}} & \textcolor{white}{\textbf{Importance}} \\
	\hline
	\endfirsthead
	
	\hline
	\rowcolor{darkblue}
	\textcolor{white}{\textbf{No.}} & \textcolor{white}{\textbf{DEB Stylized Fact}} & \textcolor{white}{\textbf{Accuracy}} & \textcolor{white}{\textbf{I have an explanation}} & \textcolor{white}{\textbf{Importance}} \\
	\hline
	\endhead
	
	\hline
	\endfoot
	
	401 & In unfavorable conditions, energy is primarily used for maintenance, not growth or reproduction. & & & \\
	\hline
	402 & Growth and reproduction rates highly depend on food availability and temperature, but modeled through the same basic parameters. & & & \\
	\hline
	403 & Nutrient assimilation follows a type II functional response (Monod or Holling function). & & & \\
	\hline
	404 & Temperature affects all reaction rates through Arrhenius relationship. & & & \\
	\hline
	405 & Acclimation changes DEB parameters systematically and predictably. & & & \\
	\hline
	406 & Seasonal variability affects energy allocation between growth, maintenance, and reproduction. & & & \\
	\hline
	407 & Nutritional quality of food affects assimilation efficiency and chemical composition of reserves. & & & \\
	\hline
	408 & Stress factors increase maintenance costs and reduce energy available for growth. & & & \\
	\hline
	409 & Photoperiod can act as a signal for changes in metabolic priorities. & & & \\
	\hline
	410 & Toxins affect DEB parameters through increased maintenance costs or reduced assimilation. & & & \\
	\hline
	411 & Oxygen availability limits metabolic rate and energy allocation efficiency in hypoxic environments. & & & \\
	\hline
	412 & pH changes affect metabolic enzyme function and energy conversion efficiency. & & & \\
	\hline
	413 & Salinity stress increases osmoregulation costs and reduces energy available for other functions. & & & \\
	\hline
	414 & Altitude adaptation involves metabolic adjustments to oxygen availability and temperature changes. & & & \\
	\hline
	415 & Humidity affects water balance and thermoregulation energy costs in terrestrial organisms. & & & \\
	\hline
	416 & Pressure changes in aquatic environments affect metabolic rate and gas exchange efficiency. & & & \\
	\hline
	417 & Light intensity affects photosynthesis-dependent organisms and circadian metabolic cycles. & & & \\
	\hline
	418 & Noise pollution increases stress hormone production and metabolic maintenance costs. & & & \\
	\hline
	419 & Chemical pollution disrupts metabolic pathways and increases detoxification energy requirements. & & & \\
	\hline
	420 & Habitat fragmentation increases movement costs and affects foraging efficiency. & & & \\
	\hline
	421 & Microclimate variation creates spatial differences in metabolic rate and energy allocation patterns. & & & \\
	\hline
	422 & Extreme weather events cause temporary metabolic disruption and increased survival energy costs. & & & \\
	\hline
	423 & Seasonal temperature cycling requires metabolic adjustment and seasonal energy storage strategies. & & & \\
	\hline
	424 & Food web position affects exposure to environmental contaminants and metabolic stress. & & & \\
	\hline
	425 & Predation pressure increases vigilance costs and affects foraging metabolic efficiency. & & & \\
	\hline
	426 & Competition intensity affects resource access and metabolic energy allocation strategies. & & & \\
	\hline
	427 & Parasite load increases maintenance costs through immune system activation and tissue damage. & & & \\
	\hline
	428 & Symbiotic relationships can reduce metabolic costs through improved resource utilization efficiency. & & & \\
	\hline
	429 & Urban environments create novel metabolic challenges through pollution and habitat modification. & & & \\
	\hline
	430 & Agricultural landscapes affect metabolic patterns through pesticide exposure and habitat simplification. & & & \\
	\hline
	431 & Climate change alters metabolic rate-temperature relationships and seasonal energy allocation patterns. & & & \\
	\hline
	432 & Ocean acidification affects calcification costs and metabolic efficiency in marine organisms. & & & \\
	\hline
	433 & Eutrophication changes food web structure and affects metabolic energy allocation strategies. & & & \\
	\hline
	434 & Invasive species alter competitive environments and metabolic resource allocation patterns. & & & \\
	\hline
	435 & Habitat restoration affects metabolic recovery through improved resource availability and reduced stress. & & & \\
	\hline
	436 & Pollution gradients create spatial variation in metabolic costs and energy allocation efficiency. & & & \\
	\hline
	437 & Edge effects in fragmented habitats affect microclimate and metabolic rate regulation. & & & \\
	\hline
	438 & Thermal refuge availability affects metabolic heat stress and energy allocation strategies. & & & \\
	\hline
	439 & Water availability limits metabolic function and energy allocation in arid environments. & & & \\
	\hline
	440 & Soil quality affects nutrient availability and metabolic efficiency in soil-dwelling organisms. & & & \\
	\hline
	441 & Fire regimes affect habitat quality and long-term metabolic adaptation strategies. & & & \\
	\hline
	442 & Flood cycles create periodic metabolic stress and require adaptive energy allocation responses. & & & \\
	\hline
	443 & Drought conditions force metabolic depression and emergency energy conservation strategies. & & & \\
	\hline
	444 & Storm frequency affects metabolic energy allocation through increased shelter and recovery costs. & & & \\
	\hline
	445 & Seasonal migration requires metabolic preparation and energy storage for long-distance movement. & & & \\
	\hline
	446 & Hibernation involves dramatic metabolic depression and specialized energy allocation patterns. & & & \\
	\hline
	447 & Estivation represents metabolic adaptation to hot, dry conditions with reduced energy expenditure. & & & \\
	\hline
	448 & Diapause mechanisms suspend metabolic activity during unfavorable environmental conditions. & & & \\
	\hline
	449 & Metabolic flexibility allows switching between different energy sources based on availability. & & & \\
	\hline
	450 & Environmental metabolic monitoring provides early indicators of ecosystem health and function. & & & \\
	\hline
	451 & Metabolic biomarkers reflect environmental stress exposure and physiological condition assessment. & & & \\
	\hline
	452 & Adaptive metabolic responses to environmental change involve genetic and phenotypic adjustments. & & & \\
	\hline
	453 & Environmental metabolic thresholds determine survival limits and distribution boundaries. & & & \\
	\hline
	454 & Metabolic niche specialization reflects adaptation to specific environmental conditions and resources. & & & \\
	\hline
	455 & Environmental metabolic gradients create selection pressures affecting population evolution. & & & \\
	\hline
	456 & Metabolic trait plasticity enables survival across variable environmental conditions. & & & \\
	\hline
	457 & Environmental metabolic synchronization coordinates population-level responses to seasonal changes. & & & \\
	\hline
	458 & Metabolic environmental tolerance determines species range limits and distribution patterns. & & & \\
	\hline
	459 & Anthropogenic environmental change affects metabolic adaptation and evolutionary responses. & & & \\
	\hline
	460 & Environmental metabolic stress affects reproductive success and population dynamics. & & & \\
	\hline
	461 & Metabolic environmental interaction complexity increases with ecosystem diversity and connectivity. & & & \\
	\hline
	462 & Environmental metabolic adaptation time lags affect species responses to rapid environmental change. & & & \\
	\hline
	463 & Metabolic environmental sensing mechanisms trigger appropriate physiological and behavioral responses. & & & \\
	\hline
	464 & Environmental metabolic trade-offs constrain optimal energy allocation under varying conditions. & & & \\
	\hline
	465 & Metabolic environmental memory effects influence responses to previously experienced conditions. & & & \\
	\hline
	466 & Environmental metabolic forecasting enables prediction of organism responses to future conditions. & & & \\
	\hline
	467 & Metabolic environmental interaction modeling requires integration of multiple environmental factors. & & & \\
	\hline
	468 & Environmental metabolic restoration involves returning metabolic function to reference conditions. & & & \\
	\hline
	469 & Metabolic environmental management requires understanding of organism-environment energy relationships. & & & \\
	\hline
	470 & Environmental metabolic conservation focuses on protecting habitats supporting optimal metabolic function. & & & \\
	\hline
	471 & Metabolic environmental education promotes understanding of organism-environment energy interactions. & & & \\
	\hline
	472 & Environmental metabolic research advances understanding of organism adaptation and survival strategies. & & & \\
	\hline
	473 & Metabolic environmental policy requires consideration of organism energy needs and environmental quality. & & & \\
	\hline
	474 & Environmental metabolic monitoring programs track changes in organism-environment energy relationships. & & & \\
	\hline
	475 & Metabolic environmental restoration success depends on reestablishing proper energy flow patterns. & & & \\
	\hline
	476 & Environmental metabolic sustainability requires maintaining conditions supporting healthy metabolic function. & & & \\
	\hline
	477 & Metabolic environmental change adaptation involves both short-term adjustments and long-term evolution. & & & \\
	\hline
	478 & Environmental metabolic stress mitigation requires reducing environmental factors that increase metabolic costs. & & & \\
	\hline
	479 & Metabolic environmental interaction research reveals fundamental principles of organism-environment relationships. & & & \\
	\hline
	480 & Environmental metabolic health assessment requires evaluation of multiple environmental stressors and metabolic responses. & & & \\
	\hline
	481 & Metabolic environmental protection mechanisms evolve to cope with natural environmental variability. & & & \\
	\hline
	482 & Environmental metabolic disruption recovery requires time for metabolic system restoration and adaptation. & & & \\
	\hline
	483 & Metabolic environmental optimization involves finding conditions supporting maximum energy allocation efficiency. & & & \\
	\hline
	484 & Environmental metabolic stability requires maintaining environmental conditions within metabolic tolerance ranges. & & & \\
	\hline
	485 & Metabolic environmental change prediction enables proactive conservation and management strategies. & & & \\
	\hline
	486 & Environmental metabolic interaction complexity challenges require interdisciplinary research approaches. & & & \\
	\hline
	487 & Metabolic environmental adaptation limits constrain species responses to rapid environmental change. & & & \\
	\hline
	488 & Environmental metabolic function maintenance requires preserving key environmental conditions and processes. & & & \\
	\hline
	489 & Metabolic environmental relationship understanding advances through long-term monitoring and experimental studies. & & & \\
	\hline
	490 & Environmental metabolic system resilience depends on maintaining diversity and connectivity in environmental conditions. & & & \\
	\hline
	491 & Metabolic environmental change resistance varies among species based on physiological flexibility and adaptation capacity. & & & \\
	\hline
	492 & Environmental metabolic impact assessment requires understanding of cumulative effects on organism energy allocation. & & & \\
	\hline
	493 & Metabolic environmental restoration goals should include reestablishing natural energy allocation patterns. & & & \\
	\hline
	494 & Environmental metabolic monitoring indicators should reflect both environmental conditions and organism metabolic responses. & & & \\
	\hline
	495 & Metabolic environmental interaction models should incorporate both direct and indirect environmental effects. & & & \\
	\hline
	496 & Environmental metabolic adaptation research reveals mechanisms of organism survival under changing conditions. & & & \\
	\hline
	497 & Metabolic environmental relationship conservation requires protecting both organisms and their environmental contexts. & & & \\
	\hline
	498 & Environmental metabolic system understanding requires integration of organism biology and environmental science. & & & \\
	\hline
	499 & Metabolic environmental change management requires adaptive strategies based on organism energy allocation principles. & & & \\
	\hline
	500 & Environmental metabolic interaction science provides foundation for understanding life in changing environments. & & & \\
	\hline
\end{longtable}

\newpage
% =====================================================
\section{Biological Experiments and Data Patterns}
% =====================================================
Instructions: \textbf{Accuracy} (1 to 5: \textbf{1} - fact is completely inaccurate; \textbf{5} - fact is completely accurate, additional: \textbf{0} - I have no opinion on the accuracy of this statement); \textbf{I have an explanation} (\textbf{Yes} - I can explain or try to explain the accuracy statement; \textbf{No} - I cannot explain or try to explain the accuracy statement, \textbf{empty cell} - I don't know enough about this statement); \textbf{Importance} (1 to 5: \textbf{1} - least important; \textbf{5} - most important for understanding DEB theory)

\begin{longtable}{|p{0.8cm}|p{7.5cm}|p{2.2cm}|p{2.2cm}|p{2.2cm}|}
	\hline
	\rowcolor{darkblue}
	\textcolor{white}{\textbf{No.}} & \textcolor{white}{\textbf{DEB Stylized Fact}} & \textcolor{white}{\textbf{Accuracy}} & \textcolor{white}{\textbf{I have an explanation}} & \textcolor{white}{\textbf{Importance}} \\
	\hline
	\endfirsthead
	
	\hline
	\rowcolor{darkblue}
	\textcolor{white}{\textbf{No.}} & \textcolor{white}{\textbf{DEB Stylized Fact}} & \textcolor{white}{\textbf{Accuracy}} & \textcolor{white}{\textbf{I have an explanation}} & \textcolor{white}{\textbf{Importance}} \\
	\hline
	\endhead
	
	\hline
	\endfoot
	
	501 & With growth and development curves, it is possible to predict food intake and energy allocation, which is confirmed experimentally. & & & \\
	\hline
	502 & The relationship between length and body mass is predictable and depends on the shape coefficient. & & & \\
	\hline
	503 & Organisms can be described by the same set of DEB parameters throughout their entire life cycle, without needing to change the model in different phases. & & & \\
	\hline
	504 & Respiratory quotient (RQ) is predictable from the chemical composition of reserves and structure. & & & \\
	\hline
	505 & Nitrogen waste correlates with structure catabolism, not reserves. & & & \\
	\hline
	506 & Lipid content of the body correlates with reserves, and protein with structure. & & & \\
	\hline
	507 & Condition factor (CF) reflects the state of reserves relative to structure. & & & \\
	\hline
	508 & Gonadosomatic index (GSI) follows predictable seasonal cycles in mature individuals. & & & \\
	\hline
	509 & Hepatosomatic index (HSI) correlates with energy state and food availability. & & & \\
	\hline
	510 & Experimental starvation shows the hierarchy of metabolic priorities. & & & \\
	\hline
	511 & Calorimetry measurements confirm predicted metabolic rates from DEB parameter estimates. & & & \\
	\hline
	512 & Feeding experiments validate functional response curves and assimilation efficiency predictions. & & & \\
	\hline
	513 & Growth rate measurements under different temperatures confirm Arrhenius relationship predictions. & & & \\
	\hline
	514 & Body composition analysis validates reserve and structure compartment predictions. & & & \\
	\hline
	515 & Reproduction timing experiments confirm maturation energy threshold hypotheses. & & & \\
	\hline
	516 & Metabolic chamber studies validate oxygen consumption and carbon dioxide production predictions. & & & \\
	\hline
	517 & Isotope labeling experiments trace energy allocation pathways predicted by DEB theory. & & & \\
	\hline
	518 & Life table experiments confirm survivorship predictions under different environmental conditions. & & & \\
	\hline
	519 & Food restriction experiments validate energy allocation hierarchy predictions. & & & \\
	\hline
	520 & Temperature acclimation studies confirm predicted changes in DEB parameters. & & & \\
	\hline
	521 & Biochemical analysis validates predicted relationships between metabolites and energy states. & & & \\
	\hline
	522 & Photoperiod manipulation experiments confirm predicted effects on reproductive timing. & & & \\
	\hline
	523 & Exercise physiology studies validate predicted relationships between activity and metabolism. & & & \\
	\hline
	524 & Toxicology experiments confirm predicted effects of contaminants on DEB parameters. & & & \\
	\hline
	525 & Field metabolic rate measurements validate laboratory-based DEB parameter estimates. & & & \\
	\hline
	526 & Genetic manipulation experiments reveal molecular basis of DEB parameter variation. & & & \\
	\hline
	527 & Comparative studies across species validate universal DEB principles and parameter relationships. & & & \\
	\hline
	528 & Longitudinal studies confirm predicted individual growth and reproduction trajectories. & & & \\
	\hline
	529 & Stress response experiments validate predicted changes in energy allocation under adverse conditions. & & & \\
	\hline
	530 & Nutritional experiments confirm predicted effects of food quality on DEB parameters. & & & \\
	\hline
	531 & Aging studies validate predicted changes in maintenance costs and reproductive output over time. & & & \\
	\hline
	532 & Population experiments confirm individual-based predictions of population dynamics. & & & \\
	\hline
	533 & Seasonal monitoring validates predicted cyclical changes in energy allocation patterns. & & & \\
	\hline
	534 & Molecular marker studies confirm predicted relationships between gene expression and metabolic state. & & & \\
	\hline
	535 & Biomarker validation studies confirm predictive power of morphological and physiological indices. & & & \\
	\hline
	536 & Multi-generation experiments validate predicted effects of parental condition on offspring performance. & & & \\
	\hline
	537 & Competition experiments confirm predicted effects of density on individual DEB parameters. & & & \\
	\hline
	538 & Migration studies validate predicted energy requirements and allocation for long-distance movement. & & & \\
	\hline
	539 & Hibernation experiments confirm predicted metabolic depression and energy conservation strategies. & & & \\
	\hline
	540 & Developmental biology experiments validate predicted maturation energy accumulation patterns. & & & \\
	\hline
	541 & Enzyme activity measurements confirm predicted relationships between metabolic capacity and DEB parameters. & & & \\
	\hline
	542 & Hormone level measurements validate predicted endocrine regulation of DEB processes. & & & \\
	\hline
	543 & Behavioral studies confirm predicted relationships between activity patterns and energy allocation. & & & \\
	\hline
	544 & Immune function experiments validate predicted energy costs of pathogen resistance and response. & & & \\
	\hline
	545 & Wound healing studies confirm predicted energy allocation for repair and maintenance processes. & & & \\
	\hline
	546 & Reproductive success experiments validate predicted optimization of offspring size and number. & & & \\
	\hline
	547 & Environmental manipulation experiments confirm predicted plasticity in DEB parameter responses. & & & \\
	\hline
	548 & Bioenergetic modeling validation confirms accuracy of DEB predictions across life stages. & & & \\
	\hline
	549 & Microbiome studies validate predicted effects of gut bacteria on assimilation efficiency. & & & \\
	\hline
	550 & Ecophysiology experiments confirm predicted adaptation to extreme environmental conditions. & & & \\
	\hline
	551 & Cellular respiration studies validate predicted energy conversion efficiency at cellular level. & & & \\
	\hline
	552 & Proteomics analysis confirms predicted protein allocation patterns in different metabolic states. & & & \\
	\hline
	553 & Metabolomics studies validate predicted metabolite profiles under different energy allocation scenarios. & & & \\
	\hline
	554 & Genomics research reveals genetic basis of DEB parameter variation among individuals and species. & & & \\
	\hline
	555 & Transcriptomics analysis confirms predicted gene expression patterns associated with energy allocation. & & & \\
	\hline
	556 & Epigenetic studies validate predicted effects of environmental conditions on developmental programming. & & & \\
	\hline
	557 & Biomechanics experiments confirm predicted relationships between structure and functional performance. & & & \\
	\hline
	558 & Sensory physiology studies validate predicted energy costs of information processing and response. & & & \\
	\hline
	559 & Neurobiological experiments confirm predicted energy allocation for brain function and maintenance. & & & \\
	\hline
	560 & Cardiovascular studies validate predicted relationships between circulation and metabolic rate. & & & \\
	\hline
	561 & Respiratory physiology experiments confirm predicted gas exchange efficiency and metabolic constraints. & & & \\
	\hline
	562 & Osmoregulation studies validate predicted energy costs of water and ion balance maintenance. & & & \\
	\hline
	563 & Thermoregulation experiments confirm predicted energy allocation for temperature control. & & & \\
	\hline
	564 & Digestive physiology studies validate predicted assimilation efficiency and gut function relationships. & & & \\
	\hline
	565 & Excretion studies confirm predicted nitrogen waste production and energy allocation patterns. & & & \\
	\hline
	566 & Muscle physiology experiments validate predicted relationships between locomotion costs and metabolism. & & & \\
	\hline
	567 & Skeletal studies confirm predicted structural investment and maintenance cost relationships. & & & \\
	\hline
	568 & Integumentary system studies validate predicted energy allocation for protection and barrier function. & & & \\
	\hline
	569 & Reproductive physiology experiments confirm predicted energy costs of gamete production and mating. & & & \\
	\hline
	570 & Developmental abnormality studies validate predicted effects of energy limitation on morphogenesis. & & & \\
	\hline
	571 & Regeneration experiments confirm predicted energy allocation for tissue repair and regrowth. & & & \\
	\hline
	572 & Cell culture studies validate predicted energy allocation patterns at cellular level. & & & \\
	\hline
	573 & Organ culture experiments confirm predicted tissue-specific metabolic characteristics. & & & \\
	\hline
	574 & Biochemical pathway analysis validates predicted metabolic flux distributions. & & & \\
	\hline
	575 & Pharmacological studies confirm predicted effects of metabolic modulators on DEB parameters. & & & \\
	\hline
	576 & Surgical intervention experiments validate predicted compensation mechanisms in DEB systems. & & & \\
	\hline
	577 & Disease model studies confirm predicted effects of pathological conditions on energy allocation. & & & \\
	\hline
	578 & Recovery experiments validate predicted healing and restoration patterns following stress. & & & \\
	\hline
	579 & Adaptation studies confirm predicted evolutionary responses to long-term environmental change. & & & \\
	\hline
	580 & Artificial selection experiments validate predicted heritability of DEB-related traits. & & & \\
	\hline
	581 & Cross-breeding studies confirm predicted genetic basis of DEB parameter differences. & & & \\
	\hline
	582 & Quantitative genetics analysis validates predicted inheritance patterns of energy allocation traits. & & & \\
	\hline
	583 & Phylogenetic analysis confirms predicted evolutionary conservation of fundamental DEB principles. & & & \\
	\hline
	584 & Biogeography studies validate predicted relationships between DEB parameters and geographic distribution. & & & \\
	\hline
	585 & Paleobiology research confirms predicted DEB principles in extinct species and evolutionary history. & & & \\
	\hline
	586 & Conservation biology applications validate practical utility of DEB models for species management. & & & \\
	\hline
	587 & Aquaculture studies confirm predicted optimization of growth and reproduction in managed systems. & & & \\
	\hline
	588 & Agricultural applications validate DEB principles in crop and livestock production systems. & & & \\
	\hline
	589 & Biomedical research confirms predicted metabolic principles relevant to human health and disease. & & & \\
	\hline
	590 & Environmental monitoring programs validate DEB biomarkers as indicators of ecosystem health. & & & \\
	\hline
	591 & Climate change research confirms predicted species responses to altered environmental conditions. & & & \\
	\hline
	592 & Pollution studies validate predicted effects of contaminants on organism energy allocation. & & & \\
	\hline
	593 & Restoration ecology applications confirm predicted recovery patterns following habitat rehabilitation. & & & \\
	\hline
	594 & Invasion biology studies validate predicted establishment success based on DEB characteristics. & & & \\
	\hline
	595 & Fisheries science applications confirm predicted sustainable harvest levels based on DEB models. & & & \\
	\hline
	596 & Wildlife management programs validate DEB-based habitat quality assessment and improvement strategies. & & & \\
	\hline
	597 & Captive breeding programs confirm predicted optimization of breeding success through DEB management. & & & \\
	\hline
	598 & Ecological risk assessment validates DEB models for predicting population-level effects of stressors. & & & \\
	\hline
	599 & Biotechnology applications confirm predicted optimization of organism performance for industrial purposes. & & & \\
	\hline
	600 & Future experimental directions will continue expanding validation of DEB theory across biological scales. & & & \\
	\hline
\end{longtable}

\newpage	
% =====================================================
\section{Dynamic Properties and Equilibrium States}
% =====================================================
Instructions: \textbf{Accuracy} (1 to 5: \textbf{1} - fact is completely inaccurate; \textbf{5} - fact is completely accurate, additional: \textbf{0} - I have no opinion on the accuracy of this statement); \textbf{I have an explanation} (\textbf{Yes} - I can explain or try to explain the accuracy statement; \textbf{No} - I cannot explain or try to explain the accuracy statement, \textbf{empty cell} - I don't know enough about this statement); \textbf{Importance} (1 to 5: \textbf{1} - least important; \textbf{5} - most important for understanding DEB theory)

\begin{longtable}{|p{0.8cm}|p{7.5cm}|p{2.2cm}|p{2.2cm}|p{2.2cm}|}
	\hline
	\rowcolor{darkblue}
	\textcolor{white}{\textbf{No.}} & \textcolor{white}{\textbf{DEB Stylized Fact}} & \textcolor{white}{\textbf{Accuracy}} & \textcolor{white}{\textbf{I have an explanation}} & \textcolor{white}{\textbf{Importance}} \\
	\hline
	\endfirsthead
	
	\hline
	\rowcolor{darkblue}
	\textcolor{white}{\textbf{No.}} & \textcolor{white}{\textbf{DEB Stylized Fact}} & \textcolor{white}{\textbf{Accuracy}} & \textcolor{white}{\textbf{I have an explanation}} & \textcolor{white}{\textbf{Importance}} \\
	\hline
	\endhead
	
	\hline
	\endfoot
	
	601 & All DEB variables tend toward equilibrium values under constant conditions. & & & \\
	\hline
	602 & Environmental perturbations cause transient responses that follow predictable patterns. & & & \\
	\hline
	603 & Reserve homeostasis is maintained by active regulatory processes. & & & \\
	\hline
	604 & Critical mass for reproduction depends on DEB parameters and environmental conditions. & & & \\
	\hline
	605 & Energy balance determines the direction of changes in body mass and reproductive state. & & & \\
	\hline
	606 & Metabolic depression is an adaptive strategy for surviving unfavorable conditions. & & & \\
	\hline
	607 & Recovery after stress follows exponential patterns of return to reference state. & & & \\
	\hline
	608 & Plasticity in DEB parameters enables adaptation to variable conditions. & & & \\
	\hline
	609 & Hysteresis in response to cyclical conditions reflects memory effects of reserves. & & & \\
	\hline
	610 & Threshold effects exist for critical functions like reproduction and survival. & & & \\
	\hline
	611 & Equilibrium body size represents balance between energy intake capacity and maintenance costs. & & & \\
	\hline
	612 & Steady-state reserve levels reflect optimal balance between energy storage and utilization. & & & \\
	\hline
	613 & Dynamic equilibrium maintains stable energy allocation ratios under constant environmental conditions. & & & \\
	\hline
	614 & Perturbation recovery time depends on the magnitude of disturbance and system resilience. & & & \\
	\hline
	615 & Multiple equilibrium states can exist under different environmental conditions. & & & \\
	\hline
	616 & Stability analysis reveals conditions under which DEB systems maintain homeostasis. & & & \\
	\hline
	617 & Oscillatory dynamics emerge from time delays in energy allocation feedback loops. & & & \\
	\hline
	618 & Equilibrium shifts occur when environmental changes exceed system adaptation capacity. & & & \\
	\hline
	619 & Transient dynamics reveal system properties not apparent at equilibrium. & & & \\
	\hline
	620 & Bifurcation points mark critical transitions between different dynamic regimes. & & & \\
	\hline
	621 & Attractor states represent stable configurations of energy allocation patterns. & & & \\
	\hline
	622 & Basin of attraction determines range of initial conditions leading to specific equilibria. & & & \\
	\hline
	623 & Saddle points represent unstable equilibria that separate different attraction basins. & & & \\
	\hline
	624 & Limit cycles emerge from nonlinear interactions in energy allocation systems. & & & \\
	\hline
	625 & Chaos can arise in DEB systems under specific parameter combinations and forcing conditions. & & & \\
	\hline
	626 & Strange attractors represent complex dynamic patterns in energy allocation behavior. & & & \\
	\hline
	627 & Sensitivity to initial conditions affects predictability of long-term system behavior. & & & \\
	\hline
	628 & Phase space analysis reveals geometric structure of DEB system dynamics. & & & \\
	\hline
	629 & Invariant manifolds organize flow patterns in multi-dimensional energy allocation space. & & & \\
	\hline
	630 & Poincaré maps reduce continuous dynamics to discrete mappings for analysis. & & & \\
	\hline
	631 & Lyapunov exponents quantify rates of divergence in nearby system trajectories. & & & \\
	\hline
	632 & Fractal dimensions characterize geometric complexity of strange attractors. & & & \\
	\hline
	633 & Return maps reveal underlying structure in apparently irregular time series. & & & \\
	\hline
	634 & Embedding dimension determines minimum space needed to reconstruct system dynamics. & & & \\
	\hline
	635 & Correlation dimension measures information content in dynamic patterns. & & & \\
	\hline
	636 & Entropy measures quantify randomness and predictability in energy allocation dynamics. & & & \\
	\hline
	637 & Reconstruction techniques recover dynamics from incomplete observational data. & & & \\
	\hline
	638 & Control theory applications enable manipulation of dynamic system behavior. & & & \\
	\hline
	639 & Optimization principles explain emergence of observed dynamic patterns. & & & \\
	\hline
	640 & Game theory describes evolutionary dynamics of energy allocation strategies. & & & \\
	\hline
	641 & Catastrophe theory analyzes sudden transitions in system behavior. & & & \\
	\hline
	642 & Network theory reveals connectivity patterns affecting system dynamics. & & & \\
	\hline
	643 & Information theory quantifies communication and coordination in DEB systems. & & & \\
	\hline
	644 & Thermodynamic principles constrain possible dynamic behaviors in energy systems. & & & \\
	\hline
	645 & Statistical mechanics explains emergence of macroscopic patterns from microscopic interactions. & & & \\
	\hline
	646 & Stochastic processes account for random fluctuations in energy allocation dynamics. & & & \\
	\hline
	647 & Noise-induced transitions can drive systems between different equilibrium states. & & & \\
	\hline
	648 & Resonance phenomena amplify periodic forcing effects on system dynamics. & & & \\
	\hline
	649 & Synchronization emerges from coupling between individual DEB systems. & & & \\
	\hline
	650 & Self-organization produces ordered patterns from initially random configurations. & & & \\
	\hline
	651 & Emergence creates system properties not present in individual components. & & & \\
	\hline
	652 & Scaling laws connect dynamics across different temporal and spatial scales. & & & \\
	\hline
	653 & Universality reveals common dynamic principles across diverse biological systems. & & & \\
	\hline
	654 & Critical phenomena occur near transition points between different phases. & & & \\
	\hline
	655 & Power laws characterize distributions and correlations in complex dynamic systems. & & & \\
	\hline
	656 & Long-range correlations persist across multiple time scales in energy allocation. & & & \\
	\hline
	657 & Memory effects influence current system behavior based on past states. & & & \\
	\hline
	658 & Adaptation mechanisms modify system parameters to improve performance. & & & \\
	\hline
	659 & Learning processes adjust energy allocation strategies based on experience. & & & \\
	\hline
	660 & Evolution shapes dynamic properties through selection on system characteristics. & & & \\
	\hline
	661 & Coevolution creates coupled dynamics between interacting species. & & & \\
	\hline
	662 & Ecological dynamics emerge from interactions between individual DEB systems. & & & \\
	\hline
	663 & Population cycles result from nonlinear feedback in resource-consumer dynamics. & & & \\
	\hline
	664 & Spatial patterns emerge from local interactions and dispersal processes. & & & \\
	\hline
	665 & Metapopulation dynamics arise from coupling between spatially separated populations. & & & \\
	\hline
	666 & Landscape effects modify local dynamics through habitat connectivity patterns. & & & \\
	\hline
	667 & Climate oscillations force periodic variations in ecosystem dynamics. & & & \\
	\hline
	668 & Disturbance regimes create temporal patterns in ecological system states. & & & \\
	\hline
	669 & Succession involves predictable changes in energy allocation patterns over time. & & & \\
	\hline
	670 & Alternative stable states exist in ecosystems under different management regimes. & & & \\
	\hline
	671 & Hysteresis effects prevent immediate return to previous states after disturbance. & & & \\
	\hline
	672 & Tipping points mark irreversible transitions to degraded ecosystem states. & & & \\
	\hline
	673 & Early warning signals precede critical transitions in ecological systems. & & & \\
	\hline
	674 & Resilience measures system capacity to maintain function under stress. & & & \\
	\hline
	675 & Resistance quantifies ability to avoid change when subjected to disturbance. & & & \\
	\hline
	676 & Recovery describes rate of return to reference state after perturbation. & & & \\
	\hline
	677 & Adaptive capacity enables system modification in response to changing conditions. & & & \\
	\hline
	678 & Transformability allows fundamental system reorganization when adaptation fails. & & & \\
	\hline
	679 & Panarchy describes nested cycles of adaptive change across scales. & & & \\
	\hline
	680 & Social-ecological systems exhibit coupled human-natural dynamics. & & & \\
	\hline
	681 & Management interventions create feedback loops affecting system dynamics. & & & \\
	\hline
	682 & Policy decisions influence boundary conditions for ecological systems. & & & \\
	\hline
	683 & Economic forces drive changes in energy allocation through resource markets. & & & \\
	\hline
	684 & Cultural values affect management practices influencing system dynamics. & & & \\
	\hline
	685 & Technology changes modify human interactions with natural systems. & & & \\
	\hline
	686 & Globalization creates long-distance coupling between distant systems. & & & \\
	\hline
	687 & Urban systems exhibit unique dynamic properties from natural systems. & & & \\
	\hline
	688 & Industrial ecology applies system dynamics to human production systems. & & & \\
	\hline
	689 & Circular economy principles minimize waste in industrial energy flows. & & & \\
	\hline
	690 & Sustainability requires maintaining stable dynamics in social-ecological systems. & & & \\
	\hline
	691 & Monitoring systems track dynamic properties for adaptive management. & & & \\
	\hline
	692 & Prediction horizons limit forecasting ability in chaotic systems. & & & \\
	\hline
	693 & Scenario analysis explores alternative futures under different assumptions. & & & \\
	\hline
	694 & Uncertainty quantification addresses limitations in dynamic system models. & & & \\
	\hline
	695 & Ensemble methods improve prediction reliability through multiple model runs. & & & \\
	\hline
	696 & Machine learning techniques identify patterns in complex dynamic data. & & & \\
	\hline
	697 & Artificial intelligence assists in understanding and predicting system behavior. & & & \\
	\hline
	698 & Big data approaches reveal previously hidden dynamic patterns. & & & \\
	\hline
	699 & Citizen science contributes observations for understanding system dynamics. & & & \\
	\hline
	700 & Future research will advance understanding of complex dynamic properties in biological systems. & & & \\
	\hline
\end{longtable}

\newpage
% =====================================================
\section{Molecular and Cellular Foundations}
% =====================================================
Instructions: \textbf{Accuracy} (1 to 5: \textbf{1} - fact is completely inaccurate; \textbf{5} - fact is completely accurate, additional: \textbf{0} - I have no opinion on the accuracy of this statement); \textbf{I have an explanation} (\textbf{Yes} - I can explain or try to explain the accuracy statement; \textbf{No} - I cannot explain or try to explain the accuracy statement, \textbf{empty cell} - I don't know enough about this statement); \textbf{Importance} (1 to 5: \textbf{1} - least important; \textbf{5} - most important for understanding DEB theory)

\begin{longtable}{|p{0.8cm}|p{7.5cm}|p{2.2cm}|p{2.2cm}|p{2.2cm}|}
	\hline
	\rowcolor{darkblue}
	\textcolor{white}{\textbf{No.}} & \textcolor{white}{\textbf{DEB Stylized Fact}} & \textcolor{white}{\textbf{Accuracy}} & \textcolor{white}{\textbf{I have an explanation}} & \textcolor{white}{\textbf{Importance}} \\
	\hline
	\endfirsthead
	
	\hline
	\rowcolor{darkblue}
	\textcolor{white}{\textbf{No.}} & \textcolor{white}{\textbf{DEB Stylized Fact}} & \textcolor{white}{\textbf{Accuracy}} & \textcolor{white}{\textbf{I have an explanation}} & \textcolor{white}{\textbf{Importance}} \\
	\hline
	\endhead
	
	\hline
	\endfoot
	
	701 & Reserves are stored in specific organelles like lipid droplets and glycogen granules. & & & \\
	\hline
	702 & Structure includes cytoskeleton, membranes, and proteins that enable cellular functions. & & & \\
	\hline
	703 & Maturation energy is linked to epigenetic changes during development. & & & \\
	\hline
	704 & Catalytic properties of enzymes determine maximum rates of metabolic processes. & & & \\
	\hline
	705 & Mitochondrial function is crucial for energy metabolism and DEB dynamics. & & & \\
	\hline
	706 & Membrane permeability affects metabolite transport and energetic properties. & & & \\
	\hline
	707 & Protein turnover contributes to maintenance costs and structural renewal. & & & \\
	\hline
	708 & Oxidative stress is linked to maintenance costs and aging processes. & & & \\
	\hline
	709 & Hormonal regulation coordinates DEB processes through signaling cascades. & & & \\
	\hline
	710 & Gene expression reflects current metabolic state and energy priorities. & & & \\
	\hline
	711 & ATP synthesis efficiency determines energy conversion rates in cellular metabolism. & & & \\
	\hline
	712 & Enzyme kinetics parameters directly influence DEB metabolic rate constants. & & & \\
	\hline
	713 & Membrane composition affects energy costs of cellular transport processes. & & & \\
	\hline
	714 & DNA repair mechanisms consume energy and contribute to maintenance costs. & & & \\
	\hline
	715 & Ribosome assembly and function require significant energy investment for protein synthesis. & & & \\
	\hline
	716 & Cellular quality control systems consume energy to maintain structural integrity. & & & \\
	\hline
	717 & Ion pump activity represents major component of cellular maintenance energy costs. & & & \\
	\hline
	718 & Autophagy processes recycle cellular components and affect energy allocation efficiency. & & & \\
	\hline
	719 & Chaperone proteins require energy to maintain proper protein folding and function. & & & \\
	\hline
	720 & Calcium signaling mechanisms consume energy and coordinate metabolic responses. & & & \\
	\hline
	721 & Cytoskeletal dynamics require energy for cellular structure maintenance and reorganization. & & & \\
	\hline
	722 & Nuclear organization affects gene expression and energy allocation patterns. & & & \\
	\hline
	723 & Endoplasmic reticulum function influences protein synthesis and energy utilization. & & & \\
	\hline
	724 & Golgi apparatus processing consumes energy for cellular secretion and membrane trafficking. & & & \\
	\hline
	725 & Lysosomal degradation processes contribute to energy recycling and cellular maintenance. & & & \\
	\hline
	726 & Peroxisome function affects lipid metabolism and energy allocation patterns. & & & \\
	\hline
	727 & Cytoplasmic streaming requires energy and affects intracellular transport efficiency. & & & \\
	\hline
	728 & Cell division costs represent significant energy investment for structural growth. & & & \\
	\hline
	729 & Apoptosis mechanisms consume energy but prevent maintenance costs of damaged cells. & & & \\
	\hline
	730 & Stem cell maintenance requires specialized energy allocation for self-renewal. & & & \\
	\hline
	731 & Differentiation processes involve energy-costly reorganization of cellular structure. & & & \\
	\hline
	732 & Cell-cell communication requires energy for signal production and reception. & & & \\
	\hline
	733 & Extracellular matrix production consumes significant energy for structural development. & & & \\
	\hline
	734 & Tissue organization requires energy for coordinated cellular arrangement. & & & \\
	\hline
	735 & Organ function emerges from cellular energy allocation patterns. & & & \\
	\hline
	736 & Metabolic compartmentalization optimizes energy utilization within cells. & & & \\
	\hline
	737 & Metabolon formation creates efficient energy conversion complexes. & & & \\
	\hline
	738 & Allosteric regulation coordinates metabolic pathways with energy demands. & & & \\
	\hline
	739 & Feedback inhibition prevents wasteful energy expenditure in metabolic pathways. & & & \\
	\hline
	740 & Metabolic switches redirect energy flow based on cellular conditions. & & & \\
	\hline
	741 & Energy charge ratios regulate metabolic activity and energy allocation. & & & \\
	\hline
	742 & Redox state affects metabolic pathway activity and energy utilization. & & & \\
	\hline
	743 & pH homeostasis requires energy and affects metabolic enzyme activity. & & & \\
	\hline
	744 & Osmotic regulation consumes energy for cellular volume and pressure control. & & & \\
	\hline
	745 & Heat shock responses require energy to maintain cellular function under stress. & & & \\
	\hline
	746 & Cold adaptation involves metabolic adjustments affecting energy allocation efficiency. & & & \\
	\hline
	747 & Circadian rhythms coordinate cellular energy allocation with environmental cycles. & & & \\
	\hline
	748 & Cell cycle checkpoints ensure proper energy allocation before division. & & & \\
	\hline
	749 & Growth factor signaling regulates cellular energy allocation between maintenance and growth. & & & \\
	\hline
	750 & Stress response pathways redirect energy allocation for cellular protection. & & & \\
	\hline
	751 & Metabolic reprogramming occurs during cellular differentiation and development. & & & \\
	\hline
	752 & Cancer cells exhibit altered energy allocation patterns optimized for rapid proliferation. & & & \\
	\hline
	753 & Aging involves progressive decline in cellular energy allocation efficiency. & & & \\
	\hline
	754 & Cellular senescence represents energy allocation strategy under irreparable damage. & & & \\
	\hline
	755 & Pluripotency maintenance requires specific energy allocation patterns in stem cells. & & & \\
	\hline
	756 & Neuronal function requires specialized energy allocation for electrical activity. & & & \\
	\hline
	757 & Muscle contraction represents dramatic temporary increase in energy allocation. & & & \\
	\hline
	758 & Immune cell activation involves rapid shifts in energy allocation patterns. & & & \\
	\hline
	759 & Hepatocyte function requires high energy allocation for metabolic processing. & & & \\
	\hline
	760 & Adipocyte energy storage and release affects whole-organism energy allocation. & & & \\
	\hline
	761 & Endocrine cell secretion requires energy for hormone production and release. & & & \\
	\hline
	762 & Reproductive cell development involves specialized energy allocation for gamete production. & & & \\
	\hline
	763 & Photosynthetic machinery requires energy investment for light harvesting apparatus. & & & \\
	\hline
	764 & Chloroplast function determines energy input capacity in photosynthetic organisms. & & & \\
	\hline
	765 & Carbon fixation pathways affect energy allocation efficiency in primary producers. & & & \\
	\hline
	766 & Nitrogen fixation requires substantial energy investment in specialized bacteria. & & & \\
	\hline
	767 & Chemosynthesis provides alternative energy input pathway in specialized environments. & & & \\
	\hline
	768 & Fermentation represents low-efficiency energy allocation strategy under anaerobic conditions. & & & \\
	\hline
	769 & Respiratory chain efficiency determines cellular energy conversion capacity. & & & \\
	\hline
	770 & Glycolysis regulation affects rapid energy allocation for immediate cellular needs. & & & \\
	\hline
	771 & Citric acid cycle function coordinates energy allocation with biosynthetic demands. & & & \\
	\hline
	772 & Fatty acid oxidation provides sustained energy for long-term cellular maintenance. & & & \\
	\hline
	773 & Amino acid catabolism contributes to energy allocation during protein turnover. & & & \\
	\hline
	774 & Nucleotide metabolism affects energy allocation for DNA and RNA synthesis. & & & \\
	\hline
	775 & Vitamin cofactor availability limits metabolic pathway efficiency and energy allocation. & & & \\
	\hline
	776 & Mineral requirements affect enzyme function and energy allocation capacity. & & & \\
	\hline
	777 & Trace element deficiencies can limit cellular energy allocation efficiency. & & & \\
	\hline
	778 & Antioxidant systems require energy investment to protect cellular structures. & & & \\
	\hline
	779 & Membrane potential maintenance consumes significant cellular energy resources. & & & \\
	\hline
	780 & Active transport processes represent major component of cellular energy allocation. & & & \\
	\hline
	781 & Passive transport efficiency affects overall cellular energy allocation patterns. & & & \\
	\hline
	782 & Vesicular transport requires energy for intracellular material movement. & & & \\
	\hline
	783 & Endocytosis and exocytosis processes consume energy for cellular material exchange. & & & \\
	\hline
	784 & Protein secretion requires energy for synthesis, modification, and transport. & & & \\
	\hline
	785 & Cellular adhesion requires energy for junction formation and maintenance. & & & \\
	\hline
	786 & Cell migration consumes significant energy for motility and directional movement. & & & \\
	\hline
	787 & Cilia and flagella function require energy for cellular movement and fluid transport. & & & \\
	\hline
	788 & Contractile apparatus requires energy for cellular shape changes and force generation. & & & \\
	\hline
	789 & Cellular mechanosensing requires energy for detecting and responding to physical forces. & & & \\
	\hline
	790 & Cellular electrical activity requires energy for ion gradient maintenance and signal propagation. & & & \\
	\hline
	791 & Photoreception requires energy for light detection and signal transduction. & & & \\
	\hline
	792 & Chemoreception requires energy for molecular detection and signal processing. & & & \\
	\hline
	793 & Mechanoreception requires energy for mechanical force detection and transduction. & & & \\
	\hline
	794 & Thermoreception requires energy for temperature detection and response. & & & \\
	\hline
	795 & Cellular computation requires energy for information processing and decision making. & & & \\
	\hline
	796 & Cellular memory requires energy for information storage and retrieval mechanisms. & & & \\
	\hline
	797 & Cellular learning requires energy for adaptive modification of cellular responses. & & & \\
	\hline
	798 & Cellular adaptation requires energy for adjusting function to environmental conditions. & & & \\
	\hline
	799 & Synthetic biology applications require understanding of cellular energy allocation principles. & & & \\
	\hline
	800 & Future cellular research will continue revealing molecular foundations of DEB principles. & & & \\
	\hline
\end{longtable}

\newpage
% =====================================================
\section{Microbiological and Parasitological Aspects}
% =====================================================
Instructions: \textbf{Accuracy} (1 to 5: \textbf{1} - fact is completely inaccurate; \textbf{5} - fact is completely accurate, additional: \textbf{0} - I have no opinion on the accuracy of this statement); \textbf{I have an explanation} (\textbf{Yes} - I can explain or try to explain the accuracy statement; \textbf{No} - I cannot explain or try to explain the accuracy statement, \textbf{empty cell} - I don't know enough about this statement); \textbf{Importance} (1 to 5: \textbf{1} - least important; \textbf{5} - most important for understanding DEB theory)

\begin{longtable}{|p{0.8cm}|p{7.5cm}|p{2.2cm}|p{2.2cm}|p{2.2cm}|}
	\hline
	\rowcolor{darkblue}
	\textcolor{white}{\textbf{No.}} & \textcolor{white}{\textbf{DEB Stylized Fact}} & \textcolor{white}{\textbf{Accuracy}} & \textcolor{white}{\textbf{I have an explanation}} & \textcolor{white}{\textbf{Importance}} \\
	\hline
	\endfirsthead
	
	\hline
	\rowcolor{darkblue}
	\textcolor{white}{\textbf{No.}} & \textcolor{white}{\textbf{DEB Stylized Fact}} & \textcolor{white}{\textbf{Accuracy}} & \textcolor{white}{\textbf{I have an explanation}} & \textcolor{white}{\textbf{Importance}} \\
	\hline
	\endhead
	
	\hline
	\endfoot
	
	801 & Pathogens increase maintenance costs through immune responses and tissue damage. & & & \\
	\hline
	802 & Symbiotic relationships can change assimilation efficiency through improved digestion. & & & \\
	\hline
	803 & Gut microbiome affects the energy value of consumed food. & & & \\
	\hline
	804 & Parasites redirect energy flows toward their own needs. & & & \\
	\hline
	805 & Immune activation consumes significant amounts of energy from reserves. & & & \\
	\hline
	806 & Antimicrobial defense activity continuously consumes energy for maintenance. & & & \\
	\hline
	807 & Probiotic microorganisms can improve nutrient assimilation. & & & \\
	\hline
	808 & Infections change energy allocation between growth, maintenance, and defense. & & & \\
	\hline
	809 & Recovery from disease requires additional energy for damage repair. & & & \\
	\hline
	810 & Pathogen resistance costs energy even in the absence of infection. & & & \\
	\hline
	811 & Microbiome diversity affects host energy allocation efficiency and metabolic stability. & & & \\
	\hline
	812 & Bacterial fermentation in the gut provides additional energy sources for the host. & & & \\
	\hline
	813 & Viral infections disrupt cellular energy production and allocation mechanisms. & & & \\
	\hline
	814 & Fungal pathogens compete directly with host cells for available nutrients and energy. & & & \\
	\hline
	815 & Parasitic worms alter host digestive efficiency and nutrient absorption patterns. & & & \\
	\hline
	816 & Microbiome-host co-evolution shapes energy allocation strategies over evolutionary time. & & & \\
	\hline
	817 & Antibiotic treatment disrupts beneficial microbiome and reduces energy allocation efficiency. & & & \\
	\hline
	818 & Chronic infections lead to persistent elevation in maintenance energy costs. & & & \\
	\hline
	819 & Microbiome composition changes seasonally affecting host energy allocation patterns. & & & \\
	\hline
	820 & Parasite load correlates negatively with host reproductive success and energy investment. & & & \\
	\hline
	821 & Beneficial bacteria produce vitamins and cofactors that improve host metabolic efficiency. & & & \\
	\hline
	822 & Pathogen-induced fever increases metabolic rate and energy expenditure significantly. & & & \\
	\hline
	823 & Microbiome-derived metabolites directly influence host energy allocation decisions. & & & \\
	\hline
	824 & Parasitic infections can alter host behavior to benefit parasite transmission at energy cost. & & & \\
	\hline
	825 & Immune memory formation requires energy investment for long-term pathogen recognition. & & & \\
	\hline
	826 & Microbiome transplantation can restore normal energy allocation patterns in diseased hosts. & & & \\
	\hline
	827 & Bacterial quorum sensing affects collective energy utilization in microbial communities. & & & \\
	\hline
	828 & Pathogen virulence factors directly target host energy production and allocation systems. & & & \\
	\hline
	829 & Commensal organisms provide colonization resistance against pathogenic species. & & & \\
	\hline
	830 & Microbiome dysbiosis leads to inefficient energy allocation and metabolic dysfunction. & & & \\
	\hline
	831 & Parasitic manipulation of host energy allocation can benefit parasite reproduction at host expense. & & & \\
	\hline
	832 & Biofilm formation by bacteria affects nutrient availability and energy allocation in environments. & & & \\
	\hline
	833 & Microbial competition for resources influences community energy allocation patterns. & & & \\
	\hline
	834 & Host-microbiome energy exchanges involve complex metabolic cross-feeding relationships. & & & \\
	\hline
	835 & Pathogen drug resistance evolution requires energy allocation for resistance mechanism maintenance. & & & \\
	\hline
	836 & Microbiome-mediated drug metabolism affects pharmaceutical energy allocation impacts. & & & \\
	\hline
	837 & Parasitic life cycle completion requires precise timing of host energy exploitation. & & & \\
	\hline
	838 & Microbial biomarkers reflect host energy allocation state and health status. & & & \\
	\hline
	839 & Probiotic supplementation can optimize host energy allocation for improved performance. & & & \\
	\hline
	840 & Pathogen transmission strategies depend on host energy allocation and activity patterns. & & & \\
	\hline
	841 & Microbiome stability requires balanced energy allocation among community members. & & & \\
	\hline
	842 & Parasitic infections can synchronize with host reproductive cycles for optimal energy exploitation. & & & \\
	\hline
	843 & Microbial enzyme production affects host digestive efficiency and energy allocation. & & & \\
	\hline
	844 & Pathogen evasion of immune responses allows prolonged energy exploitation of hosts. & & & \\
	\hline
	845 & Microbiome-host communication involves energy-costly signaling molecule production. & & & \\
	\hline
	846 & Parasitic castration redirects host reproductive energy toward parasite reproduction. & & & \\
	\hline
	847 & Microbial succession patterns reflect changing energy availability and allocation opportunities. & & & \\
	\hline
	848 & Pathogen dormancy stages conserve energy during unfavorable host conditions. & & & \\
	\hline
	849 & Microbiome therapeutic manipulation requires understanding of energy allocation consequences. & & & \\
	\hline
	850 & Parasitic infection intensity correlates with magnitude of host energy allocation disruption. & & & \\
	\hline
	851 & Microbial metabolic flexibility allows adaptation to changing host energy allocation patterns. & & & \\
	\hline
	852 & Pathogen-induced apoptosis prevents host energy allocation to infected cells. & & & \\
	\hline
	853 & Microbiome research applications include optimizing energy allocation in agricultural systems. & & & \\
	\hline
	854 & Parasitic co-infections create complex interactions affecting host energy allocation strategies. & & & \\
	\hline
	855 & Microbial community assembly follows energy allocation optimization principles. & & & \\
	\hline
	856 & Pathogen surveillance systems monitor energy allocation impacts on host populations. & & & \\
	\hline
	857 & Microbiome engineering aims to optimize host energy allocation for specific purposes. & & & \\
	\hline
	858 & Parasitic adaptation to host energy allocation creates evolutionary arms races. & & & \\
	\hline
	859 & Microbial ecosystem services include enhancing host energy allocation efficiency. & & & \\
	\hline
	860 & Pathogen control strategies must consider energy allocation impacts on host and environment. & & & \\
	\hline
	861 & Microbiome conservation requires protecting energy allocation relationships in natural systems. & & & \\
	\hline
	862 & Parasitic disease management involves restoring normal host energy allocation patterns. & & & \\
	\hline
	863 & Microbial biotechnology applications optimize energy allocation for industrial purposes. & & & \\
	\hline
	864 & Pathogen emergence patterns reflect changes in host energy allocation and environmental conditions. & & & \\
	\hline
	865 & Microbiome-host co-evolution continues shaping energy allocation strategies in modern environments. & & & \\
	\hline
	866 & Parasitic infection prevention requires understanding of energy allocation vulnerability factors. & & & \\
	\hline
	867 & Microbial community resilience depends on flexible energy allocation among members. & & & \\
	\hline
	868 & Pathogen elimination strategies must restore normal energy allocation without disrupting beneficial microbes. & & & \\
	\hline
	869 & Microbiome personalized medicine considers individual energy allocation patterns and needs. & & & \\
	\hline
	870 & Parasitic infection treatment success depends on host energy allocation recovery capacity. & & & \\
	\hline
	871 & Microbial ecology principles apply to understanding energy allocation in host-associated communities. & & & \\
	\hline
	872 & Pathogen resistance management requires balanced energy allocation between treatment and sustainability. & & & \\
	\hline
	873 & Microbiome-based diagnostics reflect host energy allocation state and disease risk. & & & \\
	\hline
	874 & Parasitic infection dynamics depend on seasonal changes in host energy allocation patterns. & & & \\
	\hline
	875 & Microbial symbiosis optimization can enhance host energy allocation efficiency and fitness. & & & \\
	\hline
	876 & Pathogen spillover events reflect disruptions in natural energy allocation systems. & & & \\
	\hline
	877 & Microbiome restoration techniques aim to reestablish optimal energy allocation relationships. & & & \\
	\hline
	878 & Parasitic infection control requires integrated understanding of energy allocation across scales. & & & \\
	\hline
	879 & Microbial community modeling incorporates energy allocation principles for prediction and management. & & & \\
	\hline
	880 & Pathogen-microbiome interactions create complex energy allocation trade-offs affecting host health. & & & \\
	\hline
	881 & Microbiome diversity conservation protects energy allocation options for future applications. & & & \\
	\hline
	882 & Parasitic adaptation mechanisms reveal fundamental constraints on energy allocation systems. & & & \\
	\hline
	883 & Microbial energy allocation research advances understanding of life in complex communities. & & & \\
	\hline
	884 & Pathogen surveillance and microbiome monitoring provide early warning of energy allocation disruptions. & & & \\
	\hline
	885 & Microbiome-host energy allocation optimization represents frontier for improving health and productivity. & & & \\
	\hline
	886 & Parasitic infection impacts extend beyond individual hosts to affect community energy allocation patterns. & & & \\
	\hline
	887 & Microbial community function depends on coordinated energy allocation among diverse organisms. & & & \\
	\hline
	888 & Pathogen control and microbiome protection require balanced approaches considering energy allocation trade-offs. & & & \\
	\hline
	889 & Microbiome-host energy allocation relationships provide models for understanding broader ecological interactions. & & & \\
	\hline
	890 & Parasitic and microbial ecology integration advances understanding of energy allocation in complex biological systems. & & & \\
	\hline
	891 & Microbiome therapeutic applications optimize energy allocation for treating various diseases and conditions. & & & \\
	\hline
	892 & Pathogen evolution and microbiome co-evolution create dynamic energy allocation landscapes. & & & \\
	\hline
	893 & Microbial community assembly rules reflect energy allocation optimization under environmental constraints. & & & \\
	\hline
	894 & Parasitic infection prevention and microbiome health promotion require integrated energy allocation strategies. & & & \\
	\hline
	895 & Microbiome research frontiers include understanding energy allocation at molecular and ecosystem scales. & & & \\
	\hline
	896 & Pathogen-microbiome-host interactions represent complex energy allocation networks requiring systems approaches. & & & \\
	\hline
	897 & Microbial energy allocation principles apply to biotechnology, medicine, agriculture, and environmental management. & & & \\
	\hline
	898 & Parasitic and microbial community dynamics reflect fundamental energy allocation principles governing life. & & & \\
	\hline
	899 & Microbiome-host energy allocation optimization continues evolving through natural and artificial selection. & & & \\
	\hline
	900 & Future microbiological and parasitological research will advance understanding of energy allocation in complex biological communities. & & & \\
	\hline
\end{longtable}
	
	\newpage
% =====================================================
\section{Thermoregulation and Biophysical Constraints}
% =====================================================
Instructions: \textbf{Accuracy} (1 to 5: \textbf{1} - fact is completely inaccurate; \textbf{5} - fact is completely accurate, additional: \textbf{0} - I have no opinion on the accuracy of this statement); \textbf{I have an explanation} (\textbf{Yes} - I can explain or try to explain the accuracy statement; \textbf{No} - I cannot explain or try to explain the accuracy statement, \textbf{empty cell} - I don't know enough about this statement); \textbf{Importance} (1 to 5: \textbf{1} - least important; \textbf{5} - most important for understanding DEB theory)

\begin{longtable}{|p{0.8cm}|p{7.5cm}|p{2.2cm}|p{2.2cm}|p{2.2cm}|}
	\hline
	\rowcolor{darkblue}
	\textcolor{white}{\textbf{No.}} & \textcolor{white}{\textbf{DEB Stylized Fact}} & \textcolor{white}{\textbf{Accuracy}} & \textcolor{white}{\textbf{I have an explanation}} & \textcolor{white}{\textbf{Importance}} \\
	\hline
	\endfirsthead
	
	\hline
	\rowcolor{darkblue}
	\textcolor{white}{\textbf{No.}} & \textcolor{white}{\textbf{DEB Stylized Fact}} & \textcolor{white}{\textbf{Accuracy}} & \textcolor{white}{\textbf{I have an explanation}} & \textcolor{white}{\textbf{Importance}} \\
	\hline
	\endhead
	
	\hline
	\endfoot
	
	901 & Thermoregulation consumes significant amounts of energy in endothermic organisms. & & & \\
	\hline
	902 & Q$_{10}$ effects influence all metabolic rates in ectothermic species. & & & \\
	\hline
	903 & Critical thermal maximum/minimum temperatures determine metabolic limits. & & & \\
	\hline
	904 & Behavioral thermoregulation can save energy compared to physiological. & & & \\
	\hline
	905 & Seasonal acclimation changes thermal tolerances and metabolic rates. & & & \\
	\hline
	906 & Heat exchange surface area limits thermoregulatory capabilities. & & & \\
	\hline
	907 & Insulation (fur, feathers, fat) affects energetic costs of thermoregulation. & & & \\
	\hline
	908 & Vascularization determines efficiency of heat transport through the body. & & & \\
	\hline
	909 & Hibernation/torpor represent extreme metabolic depressions with dramatic cost reductions. & & & \\
	\hline
	910 & Fever increases basal metabolism as part of immune response. & & & \\
	\hline
	911 & Thermoneutral zone represents optimal temperature range for minimal energy allocation to thermoregulation. & & & \\
	\hline
	912 & Brown adipose tissue provides specialized energy allocation for non-shivering thermogenesis. & & & \\
	\hline
	913 & Shivering thermogenesis converts energy stored in muscles directly to heat production. & & & \\
	\hline
	914 & Countercurrent heat exchange systems conserve energy by preventing heat loss. & & & \\
	\hline
	915 & Thermal conductance varies with body size affecting energy allocation for temperature regulation. & & & \\
	\hline
	916 & Evaporative cooling requires energy allocation for water loss and can lead to dehydration. & & & \\
	\hline
	917 & Basking behavior in ectotherms reduces energy allocation needed for metabolic heat production. & & & \\
	\hline
	918 & Thermal hysteresis proteins prevent freezing and reduce energy costs in cold-adapted organisms. & & & \\
	\hline
	919 & Heat shock proteins require energy allocation for synthesis during thermal stress. & & & \\
	\hline
	920 & Thermal tolerance windows determine geographic distribution and energy allocation strategies. & & & \\
	\hline
	921 & Body size affects thermal inertia and influences energy allocation for temperature regulation. & & & \\
	\hline
	922 & Metabolic heat production efficiency varies among tissues affecting thermoregulatory energy allocation. & & & \\
	\hline
	923 & Thermal refugia seeking behavior involves energy costs for movement and habitat selection. & & & \\
	\hline
	924 & Cold-induced thermogenesis requires immediate energy allocation from available reserves. & & & \\
	\hline
	925 & Heat stress response involves energy allocation for cellular protection and repair mechanisms. & & & \\
	\hline
	926 & Thermal acclimation involves long-term energy investment in physiological adjustments. & & & \\
	\hline
	927 & Penguin huddling behavior reduces individual energy allocation for thermoregulation through cooperation. & & & \\
	\hline
	928 & Thermal imaging reveals energy allocation patterns through heat distribution analysis. & & & \\
	\hline
	929 & Circadian thermoregulation involves energy allocation that varies with daily activity patterns. & & & \\
	\hline
	930 & Thermal stress affects reproductive energy allocation by increasing maintenance costs. & & & \\
	\hline
	931 & Heat dissipation mechanisms require energy allocation for vasodilation and circulatory adjustments. & & & \\
	\hline
	932 & Thermal buffering by environment reduces energy allocation needed for active thermoregulation. & & & \\
	\hline
	933 & Age-related changes in thermoregulation affect energy allocation efficiency in older organisms. & & & \\
	\hline
	934 & Sex differences in thermoregulation reflect different energy allocation strategies between males and females. & & & \\
	\hline
	935 & Thermal sensitivity of enzyme systems determines temperature effects on energy allocation patterns. & & & \\
	\hline
	936 & Microclimate selection allows organisms to optimize energy allocation by choosing favorable thermal conditions. & & & \\
	\hline
	937 & Thermal pollution affects aquatic organism energy allocation by altering optimal temperature ranges. & & & \\
	\hline
	938 & Climate change forces adaptation in thermoregulatory energy allocation strategies. & & & \\
	\hline
	939 & Thermal physiology modeling predicts energy allocation under different temperature scenarios. & & & \\
	\hline
	940 & Insulation quality determines energy allocation efficiency for maintaining body temperature. & & & \\
	\hline
	941 & Thermal radiation exchange affects energy balance and allocation in various environments. & & & \\
	\hline
	942 & Conductive heat loss varies with substrate contact affecting energy allocation in different habitats. & & & \\
	\hline
	943 & Convective heat transfer depends on air movement affecting energy allocation for thermoregulation. & & & \\
	\hline
	944 & Thermal mass of organisms affects rate of temperature change and energy allocation responses. & & & \\
	\hline
	945 & Polar organism adaptations include specialized energy allocation for extreme cold survival. & & & \\
	\hline
	946 & Desert organism adaptations optimize energy allocation for extreme heat tolerance. & & & \\
	\hline
	947 & Aquatic thermoregulation involves different energy allocation strategies than terrestrial environments. & & & \\
	\hline
	948 & High altitude adaptations include energy allocation adjustments for low temperature and oxygen. & & & \\
	\hline
	949 & Thermal niche partitioning affects competitive energy allocation among coexisting species. & & & \\
	\hline
	950 & Thermal performance curves describe energy allocation efficiency across temperature ranges. & & & \\
	\hline
	951 & Biophysical modeling integrates energy allocation with heat transfer mechanisms. & & & \\
	\hline
	952 & Thermal ecology applications include predicting energy allocation under climate change scenarios. & & & \\
	\hline
	953 & Temperature-dependent sex determination affects energy allocation to reproductive development. & & & \\
	\hline
	954 & Thermal shock recovery requires significant energy allocation for cellular repair mechanisms. & & & \\
	\hline
	955 & Thermogenesis regulation involves complex energy allocation trade-offs between heat production and other functions. & & & \\
	\hline
	956 & Thermal adaptation evolution shapes energy allocation strategies over geological time scales. & & & \\
	\hline
	957 & Urban heat islands affect wildlife energy allocation by altering local thermal environments. & & & \\
	\hline
	958 & Thermoregulatory behavior includes energy allocation for nest construction and microhabitat modification. & & & \\
	\hline
	959 & Thermal stress biomarkers reflect energy allocation responses to temperature challenges. & & & \\
	\hline
	960 & Global warming impacts require understanding of thermoregulatory energy allocation limits. & & & \\
	\hline
	961 & Thermal physiology research advances understanding of energy allocation under temperature stress. & & & \\
	\hline
	962 & Conservation applications include protecting thermal habitat quality for optimal energy allocation. & & & \\
	\hline
	963 & Thermoregulatory energetics affect population dynamics through individual energy allocation patterns. & & & \\
	\hline
	964 & Thermal pollution mitigation requires understanding of organism energy allocation responses. & & & \\
	\hline
	965 & Captive animal management includes optimizing thermal conditions for efficient energy allocation. & & & \\
	\hline
	966 & Thermal physiology education promotes understanding of energy allocation principles. & & & \\
	\hline
	967 & Agricultural applications optimize thermal conditions for livestock energy allocation efficiency. & & & \\
	\hline
	968 & Thermal stress testing evaluates organism energy allocation capacity under extreme conditions. & & & \\
	\hline
	969 & Climate adaptation strategies must consider thermoregulatory energy allocation constraints. & & & \\
	\hline
	970 & Thermal ecology monitoring tracks changes in organism energy allocation responses to warming. & & & \\
	\hline
	971 & Biophysical constraints limit possible energy allocation strategies under thermal stress. & & & \\
	\hline
	972 & Thermoregulatory flexibility allows adaptation of energy allocation to variable thermal conditions. & & & \\
	\hline
	973 & Thermal tolerance breeding programs aim to improve energy allocation efficiency under heat stress. & & & \\
	\hline
	974 & Building design applications consider occupant thermoregulatory energy allocation needs. & & & \\
	\hline
	975 & Thermal physiology databases support comparative analysis of energy allocation strategies. & & & \\
	\hline
	976 & Ecosystem thermal dynamics affect community energy allocation patterns and food web structure. & & & \\
	\hline
	977 & Thermoregulatory evolution continues shaping energy allocation strategies in response to climate change. & & & \\
	\hline
	978 & Thermal stress management requires integrated approaches considering energy allocation across biological scales. & & & \\
	\hline
	979 & Future thermal physiology research will advance understanding of energy allocation under changing climates. & & & \\
	\hline
	980 & Thermal biology applications continue expanding across medicine, agriculture, conservation, and biotechnology fields. & & & \\
	\hline
	981 & Thermoregulatory energy allocation represents fundamental constraint affecting all aspects of organism biology. & & & \\
	\hline
	982 & Biophysical modeling approaches integrate energy allocation with environmental thermal dynamics. & & & \\
	\hline
	983 & Thermal adaptation limits constrain possible evolutionary responses to energy allocation challenges. & & & \\
	\hline
	984 & Thermoregulatory research contributes to understanding energy allocation principles across biological diversity. & & & \\
	\hline
	985 & Climate change biology requires comprehensive understanding of thermoregulatory energy allocation responses. & & & \\
	\hline
	986 & Thermal physiology education promotes appreciation for energy allocation constraints in biological systems. & & & \\
	\hline
	987 & Conservation thermal biology protects species energy allocation capabilities under changing conditions. & & & \\
	\hline
	988 & Thermoregulatory energy allocation research advances fundamental understanding of life under thermal stress. & & & \\
	\hline
	989 & Applied thermal biology optimizes energy allocation for human benefit while protecting natural systems. & & & \\
	\hline
	990 & Integrated thermoregulatory science combines energy allocation theory with empirical thermal biology research. & & & \\
	\hline
	991 & Thermal stress physiology reveals fundamental limits on energy allocation under extreme conditions. & & & \\
	\hline
	992 & Thermoregulatory energy allocation strategies continue evolving in response to global environmental change. & & & \\
	\hline
	993 & Biophysical thermal ecology provides framework for understanding energy allocation in natural systems. & & & \\
	\hline
	994 & Thermal adaptation research reveals mechanisms of energy allocation optimization under temperature stress. & & & \\
	\hline
	995 & Thermoregulatory conservation requires protecting energy allocation capabilities and thermal habitat quality. & & & \\
	\hline
	996 & Climate thermal biology advances prediction of energy allocation responses to future conditions. & & & \\
	\hline
	997 & Thermal physiology applications benefit human welfare while advancing scientific understanding. & & & \\
	\hline
	998 & Thermoregulatory energy allocation research contributes to broader understanding of biological organization. & & & \\
	\hline
	999 & Future thermal biology will continue revealing energy allocation principles governing life under thermal constraints. & & & \\
	\hline
	1000 & Thermoregulation and biophysical constraints represent fundamental aspects of energy allocation in all living systems. & & & \\
	\hline
\end{longtable}

\newpage
% =====================================================
\section{Integrative Aspects and Emergent Properties}
% =====================================================
Instructions: \textbf{Accuracy} (1 to 5: \textbf{1} - fact is completely inaccurate; \textbf{5} - fact is completely accurate, additional: \textbf{0} - I have no opinion on the accuracy of this statement); \textbf{I have an explanation} (\textbf{Yes} - I can explain or try to explain the accuracy statement; \textbf{No} - I cannot explain or try to explain the accuracy statement, \textbf{empty cell} - I don't know enough about this statement); \textbf{Importance} (1 to 5: \textbf{1} - least important; \textbf{5} - most important for understanding DEB theory)

\begin{longtable}{|p{0.8cm}|p{7.5cm}|p{2.2cm}|p{2.2cm}|p{2.2cm}|}
	\hline
	\rowcolor{darkblue}
	\textcolor{white}{\textbf{No.}} & \textcolor{white}{\textbf{DEB Stylized Fact}} & \textcolor{white}{\textbf{Accuracy}} & \textcolor{white}{\textbf{I have an explanation}} & \textcolor{white}{\textbf{Importance}} \\
	\hline
	\endfirsthead
	
	\hline
	\rowcolor{darkblue}
	\textcolor{white}{\textbf{No.}} & \textcolor{white}{\textbf{DEB Stylized Fact}} & \textcolor{white}{\textbf{Accuracy}} & \textcolor{white}{\textbf{I have an explanation}} & \textcolor{white}{\textbf{Importance}} \\
	\hline
	\endhead
	
	\hline
	\endfoot
	
	1001 & DEB theory connects processes from molecular to ecosystem levels. & & & \\
	\hline
	1002 & Population dynamics emerge from individual DEB properties and environmental interactions. & & & \\
	\hline
	1003 & Evolutionary pressures shape DEB parameters through natural selection on fitness. & & & \\
	\hline
	1004 & Trade-offs between different life functions limit evolutionary possibilities. & & & \\
	\hline
	1005 & Energy efficiency is not always optimal due to constraints and trade-offs. & & & \\
	\hline
	1006 & DEB properties determine ecological niches and competitive abilities of species. & & & \\
	\hline
	1007 & Climate change affects DEB processes through temperature and resource availability. & & & \\
	\hline
	1008 & Habitat fragmentation changes the energy landscape available to organisms. & & & \\
	\hline
	1009 & Biogeochemical cycles are linked to DEB processes through food webs. & & & \\
	\hline
	1010 & Conservation applications use DEB models to assess population trends. & & & \\
	\hline
	1011 & Systems biology approaches integrate DEB principles with molecular mechanisms. & & & \\
	\hline
	1012 & Network effects emerge from interactions among individual DEB systems. & & & \\
	\hline
	1013 & Scaling relationships connect individual energy allocation to population and community patterns. & & & \\
	\hline
	1014 & Emergent properties arise from collective behavior of interacting DEB systems. & & & \\
	\hline
	1015 & Feedback loops between individual and population levels create complex dynamics. & & & \\
	\hline
	1016 & Hierarchical organization allows different energy allocation strategies at multiple scales. & & & \\
	\hline
	1017 & Self-organization produces ordered patterns from individual energy allocation decisions. & & & \\
	\hline
	1018 & Adaptive management requires understanding of multi-scale energy allocation responses. & & & \\
	\hline
	1019 & Ecosystem services emerge from coordinated energy allocation across species and trophic levels. & & & \\
	\hline
	1020 & Resilience properties depend on diversity and flexibility in energy allocation strategies. & & & \\
	\hline
	1021 & Critical transitions occur when energy allocation systems exceed stability thresholds. & & & \\
	\hline
	1022 & Early warning signals can predict collapse in energy allocation systems. & & & \\
	\hline
	1023 & Alternative stable states exist in ecosystems with different energy allocation configurations. & & & \\
	\hline
	1024 & Regime shifts involve fundamental reorganization of energy allocation patterns. & & & \\
	\hline
	1025 & Adaptive cycles describe temporal patterns in ecosystem energy allocation organization. & & & \\
	\hline
	1026 & Panarchy theory explains nested energy allocation dynamics across scales and time. & & & \\
	\hline
	1027 & Social-ecological systems exhibit coupled energy allocation dynamics between humans and nature. & & & \\
	\hline
	1028 & Sustainability requires maintaining stable energy allocation patterns across generations. & & & \\
	\hline
	1029 & Transformation capacity enables fundamental changes in energy allocation organization when needed. & & & \\
	\hline
	1030 & Learning and adaptation modify energy allocation strategies based on experience and feedback. & & & \\
	\hline
	1031 & Innovation in energy allocation strategies can provide competitive advantages in changing environments. & & & \\
	\hline
	1032 & Cultural evolution affects human energy allocation patterns and environmental impacts. & & & \\
	\hline
	1033 & Technology development creates new possibilities and constraints for energy allocation optimization. & & & \\
	\hline
	1034 & Global change drivers interact to create novel energy allocation challenges for organisms and ecosystems. & & & \\
	\hline
	1035 & Interdisciplinary approaches are needed to understand complex energy allocation interactions across scales. & & & \\
	\hline
	1036 & Big data approaches reveal previously hidden patterns in energy allocation systems. & & & \\
	\hline
	1037 & Machine learning techniques identify complex relationships in energy allocation data. & & & \\
	\hline
	1038 & Artificial intelligence assists in predicting energy allocation responses to environmental change. & & & \\
	\hline
	1039 & Citizen science contributes observations for understanding energy allocation at landscape scales. & & & \\
	\hline
	1040 & Remote sensing technologies monitor energy allocation patterns across large spatial scales. & & & \\
	\hline
	1041 & Molecular techniques reveal genetic basis of energy allocation trait variation. & & & \\
	\hline
	1042 & Genomics approaches identify genes and pathways controlling energy allocation strategies. & & & \\
	\hline
	1043 & Epigenetics research shows how environmental conditions affect energy allocation gene expression. & & & \\
	\hline
	1044 & Proteomics analysis reveals protein networks involved in energy allocation regulation. & & & \\
	\hline
	1045 & Metabolomics studies identify metabolic signatures of different energy allocation states. & & & \\
	\hline
	1046 & Transcriptomics research shows how gene expression changes with energy allocation patterns. & & & \\
	\hline
	1047 & Systems biology integration combines multiple omics approaches to understand energy allocation. & & & \\
	\hline
	1048 & Computational modeling predicts energy allocation responses under complex scenarios. & & & \\
	\hline
	1049 & Simulation studies explore energy allocation dynamics under different assumptions and conditions. & & & \\
	\hline
	1050 & Theoretical development advances fundamental understanding of energy allocation principles. & & & \\
	\hline
	1051 & Empirical validation confirms theoretical predictions about energy allocation patterns. & & & \\
	\hline
	1052 & Experimental manipulation tests causal relationships in energy allocation systems. & & & \\
	\hline
	1053 & Comparative studies reveal universal principles and species-specific adaptations in energy allocation. & & & \\
	\hline
	1054 & Long-term studies track energy allocation changes over extended time periods. & & & \\
	\hline
	1055 & Meta-analysis synthesizes energy allocation results across multiple studies and species. & & & \\
	\hline
	1056 & Database development supports comparative analysis of energy allocation across taxa. & & & \\
	\hline
	1057 & Standardization efforts improve comparability of energy allocation measurements and models. & & & \\
	\hline
	1058 & Quality control procedures ensure reliability of energy allocation data and analyses. & & & \\
	\hline
	1059 & Uncertainty quantification addresses limitations in energy allocation predictions. & & & \\
	\hline
	1060 & Sensitivity analysis identifies key parameters affecting energy allocation model outcomes. & & & \\
	\hline
	1061 & Model validation confirms accuracy of energy allocation predictions against independent data. & & & \\
	\hline
	1062 & Scenario analysis explores energy allocation futures under different environmental conditions. & & & \\
	\hline
	1063 & Risk assessment evaluates threats to energy allocation systems from various stressors. & & & \\
	\hline
	1064 & Decision support tools help managers optimize energy allocation for conservation and sustainability. & & & \\
	\hline
	1065 & Policy applications use energy allocation science to inform environmental regulations and management. & & & \\
	\hline
	1066 & Economic valuation quantifies benefits of maintaining healthy energy allocation systems. & & & \\
	\hline
	1067 & Social science integration addresses human dimensions of energy allocation management. & & & \\
	\hline
	1068 & Stakeholder engagement ensures energy allocation research addresses societal needs and values. & & & \\
	\hline
	1069 & Communication strategies translate energy allocation science for public understanding and support. & & & \\
	\hline
	1070 & Education programs develop capacity for understanding and managing energy allocation systems. & & & \\
	\hline
	1071 & Training initiatives build expertise in energy allocation research and application methods. & & & \\
	\hline
	1072 & International collaboration advances global understanding of energy allocation patterns and processes. & & & \\
	\hline
	1073 & Scientific networks facilitate exchange of energy allocation knowledge and methods. & & & \\
	\hline
	1074 & Technology transfer brings energy allocation research results to practical applications. & & & \\
	\hline
	1075 & Innovation systems support development of new energy allocation research tools and methods. & & & \\
	\hline
	1076 & Intellectual property considerations affect sharing and application of energy allocation innovations. & & & \\
	\hline
	1077 & Ethical frameworks guide responsible conduct of energy allocation research and applications. & & & \\
	\hline
	1078 & Sustainability principles ensure energy allocation research and management benefit current and future generations. & & & \\
	\hline
	1079 & Precautionary approaches minimize risks from energy allocation interventions and applications. & & & \\
	\hline
	1080 & Adaptive management allows learning and improvement in energy allocation strategies over time. & & & \\
	\hline
	1081 & Monitoring systems track energy allocation patterns and trends for early detection of problems. & & & \\
	\hline
	1082 & Indicator development identifies key metrics for assessing energy allocation system health and function. & & & \\
	\hline
	1083 & Threshold identification determines critical points for energy allocation system stability and function. & & & \\
	\hline
	1084 & Reference conditions establish baselines for evaluating energy allocation system status and trends. & & & \\
	\hline
	1085 & Restoration targets define goals for recovering degraded energy allocation systems. & & & \\
	\hline
	1086 & Success criteria measure achievement of energy allocation management and conservation objectives. & & & \\
	\hline
	1087 & Performance evaluation assesses effectiveness of energy allocation interventions and policies. & & & \\
	\hline
	1088 & Continuous improvement processes enhance energy allocation management through learning and adaptation. & & & \\
	\hline
	1089 & Knowledge management systems organize and preserve energy allocation information for future use. & & & \\
	\hline
	1090 & Capacity building develops institutional and individual abilities for energy allocation research and management. & & & \\
	\hline
	1091 & Funding strategies support sustained investment in energy allocation research and applications. & & & \\
	\hline
	1092 & Infrastructure development provides facilities and equipment needed for energy allocation science. & & & \\
	\hline
	1093 & Human resources development trains scientists and practitioners in energy allocation fields. & & & \\
	\hline
	1094 & Career pathways support professional development in energy allocation research and applications. & & & \\
	\hline
	1095 & Recognition systems acknowledge contributions to energy allocation science and management. & & & \\
	\hline
	1096 & Publication strategies disseminate energy allocation research results to scientific and public audiences. & & & \\
	\hline
	1097 & Conference organization facilitates exchange of energy allocation knowledge and networking. & & & \\
	\hline
	1098 & Professional societies advance energy allocation fields through standards, ethics, and advocacy. & & & \\
	\hline
	1099 & Future directions in energy allocation research will address emerging challenges and opportunities. & & & \\
	\hline
	1100 & Integrative energy allocation science provides foundation for understanding and managing life in changing world. & & & \\
	\hline
\end{longtable}

\newpage
% =====================================================
\section{Toxicology and Ecotoxicology}
% =====================================================
Instructions: \textbf{Accuracy} (1 to 5: \textbf{1} - fact is completely inaccurate; \textbf{5} - fact is completely accurate, additional: \textbf{0} - I have no opinion on the accuracy of this statement); \textbf{I have an explanation} (\textbf{Yes} - I can explain or try to explain the accuracy statement; \textbf{No} - I cannot explain or try to explain the accuracy statement, \textbf{empty cell} - I don't know enough about this statement); \textbf{Importance} (1 to 5: \textbf{1} - least important; \textbf{5} - most important for understanding DEB theory)

\begin{longtable}{|p{0.8cm}|p{7.5cm}|p{2.2cm}|p{2.2cm}|p{2.2cm}|}
	\hline
	\rowcolor{darkblue}
	\textcolor{white}{\textbf{No.}} & \textcolor{white}{\textbf{DEB Stylized Fact}} & \textcolor{white}{\textbf{Accuracy}} & \textcolor{white}{\textbf{I have an explanation}} & \textcolor{white}{\textbf{Importance}} \\
	\hline
	\endfirsthead
	
	\hline
	\rowcolor{darkblue}
	\textcolor{white}{\textbf{No.}} & \textcolor{white}{\textbf{DEB Stylized Fact}} & \textcolor{white}{\textbf{Accuracy}} & \textcolor{white}{\textbf{I have an explanation}} & \textcolor{white}{\textbf{Importance}} \\
	\hline
	\endhead
	
	\hline
	\endfoot
	
	1101 & Toxins increase maintenance costs proportional to concentration in the body and toxicity. & & & \\
	\hline
	1102 & Bioaccumulation of toxins follows simple dynamics determined by uptake and elimination rates. & & & \\
	\hline
	1103 & Critical concentration of toxins in the body determines the onset of toxic effects on metabolism. & & & \\
	\hline
	1104 & Detoxification consumes energy from reserves and can compete with growth and reproduction. & & & \\
	\hline
	1105 & Toxins can block assimilation by reducing food utilization efficiency. & & & \\
	\hline
	1106 & Reproduction is most sensitive to toxins because it requires more energy than maintenance. & & & \\
	\hline
	1107 & Toxin elimination follows first-order kinetics with species-specific rates. & & & \\
	\hline
	1108 & Older organisms accumulate more toxins due to longer exposure time. & & & \\
	\hline
	1109 & Toxic effects are reversible if concentration falls below critical level. & & & \\
	\hline
	1110 & Toxin mixtures can act synergistically increasing total maintenance costs. & & & \\
	\hline
	1111 & Pulsed toxin exposure has different effects than continuous exposure. & & & \\
	\hline
	1112 & Recovery from toxic stress requires additional energy for metabolic process repair. & & & \\
	\hline
	1113 & Toxin tolerance can develop through increased detoxification capacity. & & & \\
	\hline
	1114 & Maternal transfer of toxins affects initial offspring state (E$_0$ and toxic loads). & & & \\
	\hline
	1115 & Sublethal toxin effects can delay puberty by increasing required maturation energy. & & & \\
	\hline
	1116 & Toxins can reduce fertility by affecting energy available for reproduction. & & & \\
	\hline
	1117 & Biomarkers like condition factor reflect toxic stress through changes in reserves. & & & \\
	\hline
	1118 & Combination of starvation and toxins is particularly dangerous as it reduces energy available for detoxification. & & & \\
	\hline
	1119 & Temperature affects toxicity by changing metabolism and elimination rates. & & & \\
	\hline
	1120 & Lipophilic toxins are stored in energy reserves and released during fat mobilization. & & & \\
	\hline
	1121 & Population effects of toxins can be predicted from individual DEB-tox parameters. & & & \\
	\hline
	1122 & Chronic toxicity often results in reduced reproductive output before affecting survival. & & & \\
	\hline
	1123 & Toxins change population age structure through differential sensitivity of life stages. & & & \\
	\hline
	1124 & Early life stages are most sensitive to toxins due to high energy requirements for growth. & & & \\
	\hline
	1125 & Endocrine disruptors affect hormonal signals that regulate DEB processes. & & & \\
	\hline
	1126 & Biomarkers of effect include changes in DEB parameters like growth and reproduction rates. & & & \\
	\hline
	1127 & Population recovery time depends on generation time and degree of toxic damage. & & & \\
	\hline
	1128 & Toxins can change offspring sex ratio through effects on energy allocation during embryogenesis. & & & \\
	\hline
	1129 & Multi-generational effects occur when toxins affect maternal investment in offspring. & & & \\
	\hline
	1130 & Density-dependent effects can mask toxicity through reduced competitiveness of contaminated populations. & & & \\
	\hline
	1131 & Seasonal timing of exposure affects magnitude of effect depending on life cycle. & & & \\
	\hline
	1132 & Ecosystem services can be disrupted when key species show toxic effects. & & & \\
	\hline
	1133 & Food web bioaccumulation follows predictable patterns through trophic levels. & & & \\
	\hline
	1134 & Adaptation to toxins can evolve through selection on DEB parameters linked to tolerance. & & & \\
	\hline
	1135 & Refugia enable maintenance of non-toxic populations as sources for recolonization. & & & \\
	\hline
	1136 & Mixture toxicity in the environment is more complex than laboratory single-compound studies. & & & \\
	\hline
	1137 & Ecological risk assessment uses DEB models to assess population and ecosystem effects. & & & \\
	\hline
	1138 & Bioremediation can reduce toxic burden and enable population recovery. & & & \\
	\hline
	1139 & Climate change can amplify toxic effects through temperature and chemical stress interactions. & & & \\
	\hline
	1140 & Monitoring programs should include DEB biomarkers for early detection of ecotoxicological effects. & & & \\
	\hline
	1141 & Nanomaterials exhibit unique toxicological properties affecting energy allocation patterns. & & & \\
	\hline
	1142 & Pharmaceutical pollution in environment affects non-target organisms through energy allocation disruption. & & & \\
	\hline
	1143 & Pesticide resistance evolution requires energy allocation for maintaining detoxification mechanisms. & & & \\
	\hline
	1144 & Microplastics ingestion affects energy allocation through physical gut blockage and chemical leaching. & & & \\
	\hline
	1145 & Heavy metal toxicity disrupts energy allocation through enzyme inhibition and oxidative stress. & & & \\
	\hline
	1146 & Organic pollutant bioaccumulation follows lipid content and affects energy storage efficiency. & & & \\
	\hline
	1147 & Toxic algal blooms produce biotoxins that affect energy allocation in exposed organisms. & & & \\
	\hline
	1148 & Indoor air pollution affects energy allocation in organisms through chronic low-level exposure. & & & \\
	\hline
	1149 & Agricultural chemical mixtures create complex toxicological interactions affecting energy allocation. & & & \\
	\hline
	1150 & Industrial effluents contain multiple contaminants that interact to affect energy allocation patterns. & & & \\
	\hline
	1151 & Urban pollution creates chemical gradients affecting energy allocation across metropolitan areas. & & & \\
	\hline
	1152 & Atmospheric deposition spreads contaminants globally affecting energy allocation in remote ecosystems. & & & \\
	\hline
	1153 & Groundwater contamination affects terrestrial and aquatic energy allocation through multiple pathways. & & & \\
	\hline
	1154 & Marine pollution affects energy allocation in oceanic food webs through bioaccumulation. & & & \\
	\hline
	1155 & Arctic contamination affects energy allocation in polar organisms through long-range transport. & & & \\
	\hline
	1156 & Legacy pollutants continue affecting energy allocation decades after their use ceased. & & & \\
	\hline
	1157 & Emerging contaminants pose new challenges for understanding energy allocation impacts. & & & \\
	\hline
	1158 & Personal care product chemicals affect energy allocation in aquatic organisms through wastewater. & & & \\
	\hline
	1159 & Flame retardants accumulate in organisms affecting energy allocation through endocrine disruption. & & & \\
	\hline
	1160 & Perfluorinated compounds persist in organisms affecting long-term energy allocation patterns. & & & \\
	\hline
	1161 & Hormonal contraceptives in environment affect energy allocation in fish and wildlife. & & & \\
	\hline
	1162 & Antibiotics in environment select for resistance affecting microbial energy allocation. & & & \\
	\hline
	1163 & Chemotherapy drugs in wastewater affect non-target organism energy allocation. & & & \\
	\hline
	1164 & Illicit drug contamination affects aquatic organism energy allocation through behavioral changes. & & & \\
	\hline
	1165 & Food additive exposure affects energy allocation through digestive system impacts. & & & \\
	\hline
	1166 & Packaging material chemicals leach into food affecting consumer energy allocation. & & & \\
	\hline
	1167 & Cleaning product chemicals affect energy allocation through dermal and respiratory exposure. & & & \\
	\hline
	1168 & Cosmetic ingredients affect energy allocation through absorption and bioaccumulation. & & & \\
	\hline
	1169 & Sunscreen chemicals affect aquatic energy allocation through environmental contamination. & & & \\
	\hline
	1170 & Electronic waste releases toxins affecting energy allocation in disposal site organisms. & & & \\
	\hline
	1171 & Battery chemicals contaminate soil and water affecting terrestrial energy allocation. & & & \\
	\hline
	1172 & Tire wear particles release chemicals affecting roadside organism energy allocation. & & & \\
	\hline
	1173 & Ship paint antifoulants affect marine energy allocation through water column contamination. & & & \\
	\hline
	1174 & Mining activities release multiple contaminants affecting watershed energy allocation. & & & \\
	\hline
	1175 & Oil spills create complex mixtures affecting energy allocation in impacted ecosystems. & & & \\
	\hline
	1176 & Nuclear contamination affects energy allocation through radiation exposure and stress. & & & \\
	\hline
	1177 & Chemical warfare agents affect energy allocation through multiple toxicological mechanisms. & & & \\
	\hline
	1178 & Biological weapons target energy allocation systems for maximum effect. & & & \\
	\hline
	1179 & Terrorist contamination events require rapid assessment of energy allocation impacts. & & & \\
	\hline
	1180 & Emergency response protocols must consider energy allocation effects of contaminants. & & & \\
	\hline
	1181 & Decontamination procedures affect energy allocation through additional chemical exposure. & & & \\
	\hline
	1182 & Personal protective equipment reduces contaminant exposure preserving normal energy allocation. & & & \\
	\hline
	1183 & Occupational exposure limits protect worker energy allocation from chronic chemical stress. & & & \\
	\hline
	1184 & Environmental regulations set limits protecting ecosystem energy allocation functions. & & & \\
	\hline
	1185 & Risk communication explains energy allocation impacts to support informed decision making. & & & \\
	\hline
	1186 & Toxicological testing protocols evaluate energy allocation effects for regulatory approval. & & & \\
	\hline
	1187 & Alternative testing methods reduce animal use while maintaining energy allocation assessment quality. & & & \\
	\hline
	1188 & Computational toxicology predicts energy allocation effects from chemical structure. & & & \\
	\hline
	1189 & High-throughput screening identifies chemicals affecting energy allocation pathways. & & & \\
	\hline
	1190 & Bioassay development measures energy allocation responses to chemical exposure. & & & \\
	\hline
	1191 & Field studies validate laboratory predictions of energy allocation effects. & & & \\
	\hline
	1192 & Biomonitoring programs track contaminant levels and energy allocation responses. & & & \\
	\hline
	1193 & Exposure assessment quantifies contaminant uptake affecting energy allocation. & & & \\
	\hline
	1194 & Dose-response relationships describe energy allocation effects across exposure ranges. & & & \\
	\hline
	1195 & Benchmark dose modeling establishes safe exposure levels protecting energy allocation. & & & \\
	\hline
	1196 & Species sensitivity distributions predict energy allocation protection levels. & & & \\
	\hline
	1197 & Probabilistic risk assessment incorporates uncertainty in energy allocation predictions. & & & \\
	\hline
	1198 & Cumulative risk assessment evaluates multiple contaminant effects on energy allocation. & & & \\
	\hline
	1199 & Retrospective analysis evaluates past energy allocation impacts from contamination events. & & & \\
	\hline
	1200 & Future toxicology research will advance understanding of energy allocation effects from emerging contaminants and changing environmental conditions. & & & \\
	\hline
\end{longtable}
	
\end{document}